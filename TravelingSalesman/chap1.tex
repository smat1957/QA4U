\chapter{量子ニーリングの基礎}

\cite{b1} 西森秀稔、大関真之 著「量子アニーリングの基礎」共立出版より2.6.1

\section{問題}

予め決められた地点を全て1度ずつ訪れて元の地点に戻ってくるための最短経路を探す。

例えば、A,B,C,D,Eの5つの地点があって、仮にAからスタートすると、次の地点はB,C,D,Eの4箇所の中から選ばなければならない。
Bを選んだら次はC,D,Eの3通り。その次は2通り。最終的には$4\times 3\times 2\times = 24$通りの選び方がある事になる。
訪れる地点の数が$N$地点だと、$(N-1)\times(N-2)\times\cdots\times2\times1=(N-1)!$通りになる。$N$が大きいと総当たりで経路を探すことは困難であり、巡回セールスマン問題は「NP困難問題」に分類されている。

\section{巡回セールスマン問題をイジング模型で表す}

まず、$N\times N$の表を考え、横方向に地点の名前(A,B,C,D,E)、縦方向には何番目に訪れるかを割り当てる。

\begin{table}[h]
  \centering
  \caption{表のタイトル}
  \label{tab:hogehoge}
  \begin{tabular}{c|ccccc}
          & A & B & C & D & E \\\hline
    1番目  & 1 & 0 & 0 & 0 & 0 \\
    2番目  & 0 & 0 & 1 & 0 & 0 \\
    3番目  & 0 & 1 & 0 & 0 & 0 \\
    4番目  & 0 & 0 & 0 & 1 & 0 \\
    5番目  & 0 & 0 & 0 & 0 & 1 \\\hline
    1番目  & 1 & 0 & 0 & 0 & 0 \\
  \end{tabular}
\end{table}

セールスマンが訪れる箇所には1を、そうでない所には0を置く。
上の表の例では、$A\rightarrow C\rightarrow B\rightarrow D\rightarrow E\rightarrow A$という経路に対応している。

式で表現するために、表の各箇所に対応した2値変数$q_{\alpha i}$を割り当ててセールスマンの辿る経路を表現する。$\alpha$は地点名($A,B,C,\cdots$)を、$i$は巡る順番($1,2,3,\cdots$)を表している。

地点$\alpha$と$\beta$の間の距離$d_{\alpha \beta}$が予め与えられているとすると、セールスマンが巡る全経路長$L$は、
\[
L = \sum_{\alpha,\beta}\sum_{i=1}^N d_{\alpha \beta}q_{\alpha, i}q_{\beta, i+1}
\]
$q_{\alpha,i}$は2値変数(0か1)なので、$q_{\alpha, i}$と$q_{\beta, i+1}$の両方が1の場合に限り、$d_{\alpha \beta}$が$L$に加算される事になる。この経路長$L$を最小にする$\{q_{\alpha,i}\}$(0か1か)、を選ぶことになる。

ただし、
\begin{itemize}
\item 各地点には1度しか訪れない(表では、各列に1は1つだけ)\\
$\longrightarrow$ 各$\alpha$において、$(\sum_i q_{\alpha, i} - 1)^2=0$
\item 各時点で訪れる地点は1箇所だけ(表では、各行に1は1つだけ)\\
$\longrightarrow$ 各$i$において、$(\sum_\alpha q_{\alpha, i} - 1)^2=0$
\end{itemize}
という2つの制約が課された上で、$L$が最小になる様に$\{q_{\alpha, i}\}$を選ばなければならない。

以上の考察より、目的関数全体の$H$は次の様になる
\[
H=\sum_{\alpha,\beta}\sum_i d_{\alpha\beta}q_{\alpha, i}q_{\beta, i+1} + \lambda\sum_\alpha\bigg(\sum_i q_{\alpha, i} - 1\bigg)^2 + \lambda\sum_i\bigg(\sum_\alpha q_{\alpha, i} - 1\bigg)^2
\]
$\lambda$は正の定数。

\section{式の展開}

\begin{eqnarray*}
  H &=& \lambda\sum_\alpha\bigg((\sum_i q_{\alpha, i})^2 - 2 \sum_i q_{\alpha, i}\bigg) + \lambda\sum_i\bigg((\sum_\alpha q_{\alpha, i})^2 - 2\sum_\alpha q_{\alpha, i}\bigg) + \sum_{\alpha,\beta}\sum_i d_{\alpha\beta}q_{\alpha, i}q_{\beta, i+1}\\
  &=& \lambda\sum_\alpha\bigg(\sum_i q_{\alpha,i}^2 +2\sum_{i,j}q_{\alpha,i}q_{\alpha,j}-2\sum_i q_{\alpha,i}\bigg) + \lambda\sum_i\bigg(\sum_\alpha q_{\alpha,i}^2 +2\sum_{\alpha,\beta}q_{\alpha,i}q_{\beta,i}-2\sum_\alpha q_{\alpha,i}\bigg)\\
  &+& \sum_{\alpha,\beta}\sum_i d_{\alpha\beta}q_{\alpha, i}q_{\beta, i+1}\\
&=& -\lambda\sum_\alpha\bigg(\sum_i q_{\alpha,i} - 2\sum_i\sum_j q_{\alpha,i}q_{\alpha,j}\bigg) - \lambda\sum_i\bigg(\sum_\alpha q_{\alpha,i} - 2\sum_\alpha\sum_\beta q_{\alpha,i}q_{\beta,i}\bigg) + \sum_{\alpha,\beta}\sum_i d_{\alpha\beta}q_{\alpha, i}q_{\beta, i+1}
\end{eqnarray*}

ここで$q$は2値変数なので、$q^2=q$が成り立つ。また、定数は最小化では無視できる。

\section{実装}

\lstinputlisting[caption=巡回セールスマン問題,label=p01]{tsp01.py}

\newpage

[実行結果] A,C,B,E,D が循環している(どこからスタートしたかによる)

\begin{figure}[h]
  \centering
  \begin{minipage}[c]{0.49\columnwidth}
    \centering
    \begin{verbatim}
      ['A', 'C', 'B', 'E', 'D'] 19.0
      ['E', 'B', 'C', 'A', 'D'] 19.0
      ['D', 'E', 'B', 'C', 'A'] 19.0
      ['B', 'E', 'D', 'A', 'C'] 19.0
      ['D', 'A', 'C', 'B', 'E'] 19.0
      ['C', 'B', 'E', 'D', 'A'] 19.0
      ['C', 'A', 'D', 'E', 'B'] 19.0
      ['A', 'D', 'E', 'B', 'C'] 19.0

      プロセスは終了コード 0 で終了しました
    \end{verbatim}
      %\centering
      %\includegraphics[width=0.9\columnwidth]{a.png}
      %\caption{左}
      %\label{fig:a}
  \end{minipage}
  \begin{minipage}[c]{0.49\columnwidth}
    \centering
    \begin{tabular}{|c|ccccc|ccccc|}\hline
      \multirow{2}{*}{} & \multicolumn{10}{|c|}{同じコストで、あり得る並び} \\\cline{2-11}
       & \multicolumn{5}{|c|}{順方向} & \multicolumn{5}{|c|}{逆方向} \\\hline
       & A & C & B & E & D & D & E & B & C & A \\
       & C & B & E & D & A & A & D & E & B & C \\
       & B & E & D & A & C & C & A & D & E & B \\
      $\rightarrow$ & E & D & A & C & B & B & C & A & D & E \\
       & D & A & C & B & E & E & B & C & A & D \\\hline
     \end{tabular}
     \leftline{($\rightarrow$この2つは出ていない様です)}
      %\centering
      %\includegraphics[width=0.9\columnwidth]{b.png}
      %\caption{右}
      %\label{fig:b}
  \end{minipage}
  \end{figure}