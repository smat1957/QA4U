% latex uft-8
\documentclass[uplatex,dvipdfmx,a4paper,11pt,oneside,openany]{jsbook}
%
\usepackage[dvipdfmx]{graphicx}
\usepackage{amsmath,amssymb}
\usepackage{bm}
\usepackage{physics}
\usepackage{graphicx}
\usepackage{ascmac}
\usepackage{amsmath}
\usepackage{setspace}
\usepackage{here}
\usepackage{fancybox}
\usepackage{url}
\usepackage{listings,jlisting} %日本語のコメントアウトをする場合jlistingが必要
\usepackage{xcolor}
\usepackage{comment}
\usepackage{multicol}
\usepackage{amsmath, amssymb}
\usepackage{type1cm}
\usepackage{fancybox}

\definecolor{mygreen}{rgb}{0,0.6,0}
\definecolor{mygray}{rgb}{0.5,0.5,0.5}
\definecolor{mymauve}{rgb}{0.58,0,0.82}

%\begin{comment}
\lstset{
  backgroundcolor=\color{white},   % choose the background color; you must add \usepackage{color} or \usepackage{xcolor}; should come as last argument
  basicstyle=\footnotesize,        % the size of the fonts that are used for the code
  breakatwhitespace=false,         % sets if automatic breaks should only happen at whitespace
  breaklines=true,                 % sets automatic line breaking
  captionpos=b,                    % sets the caption-position to bottom
  commentstyle=\color{mygreen},    % comment style
  deletekeywords={...},            % if you want to delete keywords from the given language
  escapeinside={\%*}{*)},          % if you want to add LaTeX within your code
  extendedchars=true,              % lets you use non-ASCII characters; for 8-bits encodings only, does not work with UTF-8
  firstnumber=1000,                % start line enumeration with line 1000
  frame=single,	                   % adds a frame around the code
  keepspaces=true,                 % keeps spaces in text, useful for keeping indentation of code (possibly needs columns=flexible)
  keywordstyle=\color{blue},       % keyword style
  language=Octave,                 % the language of the code
  morekeywords={*,...},            % if you want to add more keywords to the set
  numbers=left,                    % where to put the line-numbers; possible values are (none, left, right)
  numbersep=5pt,                   % how far the line-numbers are from the code
  numberstyle=\tiny\color{mygray}, % the style that is used for the line-numbers
  rulecolor=\color{black},         % if not set, the frame-color may be changed on line-breaks within not-black text (e.g. comments (green here))
  showspaces=false,                % show spaces everywhere adding particular underscores; it overrides 'showstringspaces'
  showstringspaces=false,          % underline spaces within strings only
  showtabs=false,                  % show tabs within strings adding particular underscores
  stepnumber=2,                    % the step between two line-numbers. If it's 1, each line will be numbered
  stringstyle=\color{mymauve},     % string literal style
  tabsize=2,	                   % sets default tabsize to 2 spaces
  title=\lstname                   % show the filename of files included with \lstinputlisting; also try caption instead of title
}
%\end{comment}

\lstdefinelanguage{mypy}{
  % リテラルと演算記号
  morekeywords=[1]{+=,=,==,!=,!,>,<,>=,<=,++,-,+,*,\%,/},
  % 予約語
  morekeywords=[2]{False,None,True,and,as,assert,async,await,break,
    class,continue,def,del,elif,else,except,finally,for,from,global,
    if,import,in,is,lambda,nonlocal,not,or,pass,raise,return,try,while,
    with,yield,print,match,case
  },
  % 識別子
  morekeywords=[3]{defined_func},
  % 区切り文字を強制的に色付け
  literate=*{.}{{\color{delimiter}.}}1
            {,}{{\color{delimiter},}}1 {:}{{\color{delimiter}:}}1
            {)}{{\color{delimiter})}}1 {(}{{\color{delimiter}(}}1
            {[}{{\color{delimiter}[}}1 {]}{{\color{delimiter}]}}1
            {\{}{{\color{delimiter}\{}}1 {\}}{{\color{delimiter}\}}}1,
  % 大文字小文字を区別
  sensitive=true,
  % 行コメントの設定
  morecomment=[l]{\#},
  % Stringリテラルの設定
  morestring=[b]{\'},
  morestring=[b]{\"},
  % 単語として扱う文字
  alsoletter={\%<>=+-*\/1234567890!},
  % 枠
  frame=none,
  % 長くなったら途中で改行
  breaklines=true,
  % 自動改行時のインデント
  breakindent=12pt,
  % 文字間隔を一定に
  columns=fixed,
  % 文字の横のサイズを小さく
  basewidth=0.5em,
  % 行番号を左に
  numbers=left,
  % 行番号の書式
  numberstyle={\scriptsize\color{white}},
  % 行番号の増加数は1=連番に
  stepnumber=1,
  % フレームの左の余白
  framexleftmargin=18pt,
  % スペースを省略せず保持
  keepspaces=true,
  % インデントサイズ
  tabsize=4,
  backgroundcolor={},                                   % 背景色=透明
  basicstyle={\small\ttfamily\color{white}},            % 通常部分の書式
  identifierstyle={\small\color{white}},                % 識別子の書式
  commentstyle={\small\color{comment}},                 % コメントの書式
  keywordstyle=[1]{\small\bfseries\color{literal}},     % リテラルと演算記号の書式
  keywordstyle=[2]{\small\bfseries\color{reserved}},    % 予約語の書式
  keywordstyle=[3]{\small\bfseries\color{identifier}},  % 自分で定義した識別子の書式
  stringstyle={\small\ttfamily\color{literal}},         % 文字列の書式
}
%ここからソースコードの表示に関する設定
\lstdefinestyle{customc}{
  belowcaptionskip=1\baselineskip,
  breaklines=true,
  numbers=left,
  frame=single,
  xleftmargin=\parindent,
  language=C,
  showstringspaces=false,
  basicstyle=\footnotesize\ttfamily,
  keywordstyle=\bfseries\color{green!40!black},
  commentstyle=\itshape\color{purple!40!black},
  identifierstyle=\color{blue},
  stringstyle=\color{orange},
}

\lstdefinestyle{custompy}{
  belowcaptionskip=1\baselineskip,
  breaklines=true,
  numbers=none,
  frame=l,
  xleftmargin=\parindent,
  language=python,
  showstringspaces=false,
  basicstyle=\footnotesize\ttfamily,
  keywordstyle=\bfseries\color{green!40!black},
  commentstyle=\itshape\color{purple!40!black},
  identifierstyle=\color{blue},
  stringstyle=\color{orange},
}

\lstdefinestyle{customasm}{
  belowcaptionskip=1\baselineskip,
  frame=L,
  xleftmargin=\parindent,
  language=[x86masm]Assembler,
  basicstyle=\footnotesize\ttfamily,
  commentstyle=\itshape\color{purple!40!black},
}

\lstset{escapechar=@,style=custompy}
\renewcommand{\lstlistingname}{プログラム}

\begin{comment}
\makeatletter
\def\ps@plainfoot{%
  \let\@mkboth\@gobbletwo
  \let\@oddhead\@empty
  \def\@oddfoot{\normalfont\hfil-- \thepage\ --\hfil}%
  \let\@evenhead\@empty
  \let\@evenfoot\@oddfoot}
  \let\ps@plain\ps@plainfoot
  \renewcommand{\chapter}{%
  \if@openright\cleardoublepage\else\clearpage\fi
  \global\@topnum\z@
  \secdef\@chapter\@schapter}
\makeatother
\end{comment}

%
\newcommand{\maru}[1]{{\ooalign{%
\hfil\hbox{$\bigcirc$}\hfil\crcr%
\hfil\hbox{#1}\hfil}}}
%
\setlength{\textwidth}{\fullwidth}
\setlength{\textheight}{40\baselineskip}
\addtolength{\textheight}{\topskip}
\setlength{\voffset}{-0.55in}
%
\makeatletter
\newenvironment{tablehere}
  {\def\@captype{table}}
  {}
\newenvironment{figurehere}
  {\def\@captype{figure}}
  {}
\makeatother
%
\title{巡回セールスマン問題(TSP:Traveling Salesman Problem)}
\author{smat1957@gmail.com\thanks{https://altema.is.tohoku.ac.jp/QA4U3/}}
\date{\today}
%
\begin{document}

\maketitle

\chapter{量子ニーリングの基礎}

\cite{b1} 西森秀稔、大関真之 著「量子アニーリングの基礎」共立出版より2.6.1

\section{問題}

予め決められた地点を全て1度ずつ訪れて元の地点に戻ってくるための最短経路を探す。

例えば、A,B,C,D,Eの5つの地点があって、仮にAからスタートすると、次の地点はB,C,D,Eの4箇所の中から選ばなければならない。
Bを選んだら次はC,D,Eの3通り。その次は2通り。最終的には$4\times 3\times 2\times = 24$通りの選び方がある事になる。
訪れる地点の数が$N$地点だと、$(N-1)\times(N-2)\times\cdots\times2\times1=(N-1)!$通りになる。$N$が大きいと総当たりで経路を探すことは困難であり、巡回セールスマン問題は「NP困難問題」に分類されている。

\section{巡回セールスマン問題をイジング模型で表す}

まず、$N\times N$の表を考え、横方向に地点の名前(A,B,C,D,E)、縦方向には何番目に訪れるかを割り当てる。

\begin{table}[h]
  \centering
  \caption{表のタイトル}
  \label{tab:hogehoge}
  \begin{tabular}{c|ccccc}
          & A & B & C & D & E \\\hline
    1番目  & 1 & 0 & 0 & 0 & 0 \\
    2番目  & 0 & 0 & 1 & 0 & 0 \\
    3番目  & 0 & 1 & 0 & 0 & 0 \\
    4番目  & 0 & 0 & 0 & 1 & 0 \\
    5番目  & 0 & 0 & 0 & 0 & 1 \\\hline
    1番目  & 1 & 0 & 0 & 0 & 0 \\
  \end{tabular}
\end{table}

セールスマンが訪れる箇所には1を、そうでない所には0を置く。
上の表の例では、$A\rightarrow C\rightarrow B\rightarrow D\rightarrow E\rightarrow A$という経路に対応している。

式で表現するために、表の各箇所に対応した2値変数$q_{\alpha i}$を割り当ててセールスマンの辿る経路を表現する。$\alpha$は地点名($A,B,C,\cdots$)を、$i$は巡る順番($1,2,3,\cdots$)を表している。

地点$\alpha$と$\beta$の間の距離$d_{\alpha \beta}$が予め与えられているとすると、セールスマンが巡る全経路長$L$は、
\[
L = \sum_{\alpha,\beta}\sum_{i=1}^N d_{\alpha \beta}q_{\alpha, i}q_{\beta, i+1}
\]
$q_{\alpha,i}$は2値変数(0か1)なので、$q_{\alpha, i}$と$q_{\beta, i+1}$の両方が1の場合に限り、$d_{\alpha \beta}$が$L$に加算される事になる。この経路長$L$を最小にする$\{q_{\alpha,i}\}$(0か1か)、を選ぶことになる。

ただし、
\begin{itemize}
\item 各地点には1度しか訪れない(表では、各列に1は1つだけ)\\
$\longrightarrow$ 各$\alpha$において、$(\sum_i q_{\alpha, i} - 1)^2=0$
\item 各時点で訪れる地点は1箇所だけ(表では、各行に1は1つだけ)\\
$\longrightarrow$ 各$i$において、$(\sum_\alpha q_{\alpha, i} - 1)^2=0$
\end{itemize}
という2つの制約が課された上で、$L$が最小になる様に$\{q_{\alpha, i}\}$を選ばなければならない。

以上の考察より、目的関数全体の$H$は次の様になる
\[
H=\sum_{\alpha,\beta}\sum_i d_{\alpha\beta}q_{\alpha, i}q_{\beta, i+1} + \lambda\sum_\alpha\bigg(\sum_i q_{\alpha, i} - 1\bigg)^2 + \lambda\sum_i\bigg(\sum_\alpha q_{\alpha, i} - 1\bigg)^2
\]
$\lambda$は正の定数。

\section{式の展開}

\begin{eqnarray*}
  H &=& \lambda\sum_\alpha\bigg((\sum_i q_{\alpha, i})^2 - 2 \sum_i q_{\alpha, i}\bigg) + \lambda\sum_i\bigg((\sum_\alpha q_{\alpha, i})^2 - 2\sum_\alpha q_{\alpha, i}\bigg) + \sum_{\alpha,\beta}\sum_i d_{\alpha\beta}q_{\alpha, i}q_{\beta, i+1}\\
  &=& \lambda\sum_\alpha\bigg(\sum_i q_{\alpha,i}^2 +2\sum_{i,j}q_{\alpha,i}q_{\alpha,j}-2\sum_i q_{\alpha,i}\bigg) + \lambda\sum_i\bigg(\sum_\alpha q_{\alpha,i}^2 +2\sum_{\alpha,\beta}q_{\alpha,i}q_{\beta,i}-2\sum_\alpha q_{\alpha,i}\bigg)\\
  &+& \sum_{\alpha,\beta}\sum_i d_{\alpha\beta}q_{\alpha, i}q_{\beta, i+1}\\
&=& -\lambda\sum_\alpha\bigg(\sum_i q_{\alpha,i} - 2\sum_i\sum_j q_{\alpha,i}q_{\alpha,j}\bigg) - \lambda\sum_i\bigg(\sum_\alpha q_{\alpha,i} - 2\sum_\alpha\sum_\beta q_{\alpha,i}q_{\beta,i}\bigg) + \sum_{\alpha,\beta}\sum_i d_{\alpha\beta}q_{\alpha, i}q_{\beta, i+1}
\end{eqnarray*}

ここで$q$は2値変数なので、$q^2=q$が成り立つ。また、定数は最小化では無視できる。

\section{実装}

\lstinputlisting[caption=巡回セールスマン問題,label=p01]{tsp01.py}

\newpage

[実行結果] A,C,B,E,D が循環している(どこからスタートしたかによる)

\begin{figure}[h]
  \centering
  \begin{minipage}[c]{0.49\columnwidth}
    \centering
    \begin{verbatim}
      ['A', 'C', 'B', 'E', 'D'] 19.0
      ['E', 'B', 'C', 'A', 'D'] 19.0
      ['D', 'E', 'B', 'C', 'A'] 19.0
      ['B', 'E', 'D', 'A', 'C'] 19.0
      ['D', 'A', 'C', 'B', 'E'] 19.0
      ['C', 'B', 'E', 'D', 'A'] 19.0
      ['C', 'A', 'D', 'E', 'B'] 19.0
      ['A', 'D', 'E', 'B', 'C'] 19.0

      プロセスは終了コード 0 で終了しました
    \end{verbatim}
      %\centering
      %\includegraphics[width=0.9\columnwidth]{a.png}
      %\caption{左}
      %\label{fig:a}
  \end{minipage}
  \begin{minipage}[c]{0.49\columnwidth}
    \centering
    \begin{tabular}{|c|ccccc|ccccc|}\hline
      \multirow{2}{*}{} & \multicolumn{10}{|c|}{同じコストで、あり得る並び} \\\cline{2-11}
       & \multicolumn{5}{|c|}{順方向} & \multicolumn{5}{|c|}{逆方向} \\\hline
       & A & C & B & E & D & D & E & B & C & A \\
       & C & B & E & D & A & A & D & E & B & C \\
       & B & E & D & A & C & C & A & D & E & B \\
      $\rightarrow$ & E & D & A & C & B & B & C & A & D & E \\
       & D & A & C & B & E & E & B & C & A & D \\\hline
     \end{tabular}
     \leftline{($\rightarrow$この2つは出ていない様です)}
      %\centering
      %\includegraphics[width=0.9\columnwidth]{b.png}
      %\caption{右}
      %\label{fig:b}
  \end{minipage}
  \end{figure}

\begin{thebibliography}{9}
  \bibitem{b1} 西森秀稔、大関真之, 量子アニーリングの基礎, 共立出版, 2.6.1
  \bibitem{b2} \url{https://qiita.com/suzuki_sh/items/32468fbbe3f400edce35}
  \bibitem{b3} \url{https://qiita.com/yabish/items/9f42e3752174aef8b79f}
  \bibitem{b4} \url{https://qiita.com/yufuji25/items/0425567b800443a679f7}
  \bibitem{b5} \url{https://motojapan.hateblo.jp/entry/2017/11/15/082738}
\end{thebibliography}

\end{document}