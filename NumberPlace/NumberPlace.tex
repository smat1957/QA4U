% latex uft-8
\documentclass[uplatex,dvipdfmx,a4paper,11pt,oneside,openany]{jsbook}
%
\usepackage[dvipdfmx]{graphicx}
\usepackage{amsmath,amssymb}
\usepackage{bm}
\usepackage{physics}
\usepackage{graphicx}
\usepackage{ascmac}
\usepackage{amsmath}
\usepackage{setspace}
\usepackage{here}
\usepackage{fancybox}
\usepackage{url}
\usepackage{listings,jlisting} %日本語のコメントアウトをする場合jlistingが必要
\usepackage{xcolor}
\usepackage{comment}
\usepackage{multicol}
\usepackage{amsmath, amssymb}
\usepackage{type1cm}
\usepackage{booktabs}
\usepackage{multirow}

\definecolor{mygreen}{rgb}{0,0.6,0}
\definecolor{mygray}{rgb}{0.5,0.5,0.5}
\definecolor{mymauve}{rgb}{0.58,0,0.82}

%\begin{comment}
\lstset{
  backgroundcolor=\color{white},   % choose the background color; you must add \usepackage{color} or \usepackage{xcolor}; should come as last argument
  basicstyle=\footnotesize,        % the size of the fonts that are used for the code
  breakatwhitespace=false,         % sets if automatic breaks should only happen at whitespace
  breaklines=true,                 % sets automatic line breaking
  captionpos=b,                    % sets the caption-position to bottom
  commentstyle=\color{mygreen},    % comment style
  deletekeywords={...},            % if you want to delete keywords from the given language
  escapeinside={\%*}{*)},          % if you want to add LaTeX within your code
  extendedchars=true,              % lets you use non-ASCII characters; for 8-bits encodings only, does not work with UTF-8
  firstnumber=1000,                % start line enumeration with line 1000
  frame=single,	                   % adds a frame around the code
  keepspaces=true,                 % keeps spaces in text, useful for keeping indentation of code (possibly needs columns=flexible)
  keywordstyle=\color{blue},       % keyword style
  language=Octave,                 % the language of the code
  morekeywords={*,...},            % if you want to add more keywords to the set
  numbers=left,                    % where to put the line-numbers; possible values are (none, left, right)
  numbersep=5pt,                   % how far the line-numbers are from the code
  numberstyle=\tiny\color{mygray}, % the style that is used for the line-numbers
  rulecolor=\color{black},         % if not set, the frame-color may be changed on line-breaks within not-black text (e.g. comments (green here))
  showspaces=false,                % show spaces everywhere adding particular underscores; it overrides 'showstringspaces'
  showstringspaces=false,          % underline spaces within strings only
  showtabs=false,                  % show tabs within strings adding particular underscores
  stepnumber=2,                    % the step between two line-numbers. If it's 1, each line will be numbered
  stringstyle=\color{mymauve},     % string literal style
  tabsize=2,	                   % sets default tabsize to 2 spaces
  title=\lstname                   % show the filename of files included with \lstinputlisting; also try caption instead of title
}
%\end{comment}

\lstdefinelanguage{mypy}{
  % リテラルと演算記号
  morekeywords=[1]{+=,=,==,!=,!,>,<,>=,<=,++,-,+,*,\%,/},
  % 予約語
  morekeywords=[2]{False,None,True,and,as,assert,async,await,break,
    class,continue,def,del,elif,else,except,finally,for,from,global,
    if,import,in,is,lambda,nonlocal,not,or,pass,raise,return,try,while,
    with,yield,print,match,case
  },
  % 識別子
  morekeywords=[3]{defined_func},
  % 区切り文字を強制的に色付け
  literate=*{.}{{\color{delimiter}.}}1
            {,}{{\color{delimiter},}}1 {:}{{\color{delimiter}:}}1
            {)}{{\color{delimiter})}}1 {(}{{\color{delimiter}(}}1
            {[}{{\color{delimiter}[}}1 {]}{{\color{delimiter}]}}1
            {\{}{{\color{delimiter}\{}}1 {\}}{{\color{delimiter}\}}}1,
  % 大文字小文字を区別
  sensitive=true,
  % 行コメントの設定
  morecomment=[l]{\#},
  % Stringリテラルの設定
  morestring=[b]{\'},
  morestring=[b]{\"},
  % 単語として扱う文字
  alsoletter={\%<>=+-*\/1234567890!},
  % 枠
  frame=none,
  % 長くなったら途中で改行
  breaklines=true,
  % 自動改行時のインデント
  breakindent=12pt,
  % 文字間隔を一定に
  columns=fixed,
  % 文字の横のサイズを小さく
  basewidth=0.5em,
  % 行番号を左に
  numbers=left,
  % 行番号の書式
  numberstyle={\scriptsize\color{white}},
  % 行番号の増加数は1=連番に
  stepnumber=1,
  % フレームの左の余白
  framexleftmargin=18pt,
  % スペースを省略せず保持
  keepspaces=true,
  % インデントサイズ
  tabsize=4,
  backgroundcolor={},                                   % 背景色=透明
  basicstyle={\small\ttfamily\color{white}},            % 通常部分の書式
  identifierstyle={\small\color{white}},                % 識別子の書式
  commentstyle={\small\color{comment}},                 % コメントの書式
  keywordstyle=[1]{\small\bfseries\color{literal}},     % リテラルと演算記号の書式
  keywordstyle=[2]{\small\bfseries\color{reserved}},    % 予約語の書式
  keywordstyle=[3]{\small\bfseries\color{identifier}},  % 自分で定義した識別子の書式
  stringstyle={\small\ttfamily\color{literal}},         % 文字列の書式
}
%ここからソースコードの表示に関する設定
\lstdefinestyle{customc}{
  belowcaptionskip=1\baselineskip,
  breaklines=true,
  numbers=left,
  frame=single,
  xleftmargin=\parindent,
  language=C,
  showstringspaces=false,
  basicstyle=\footnotesize\ttfamily,
  keywordstyle=\bfseries\color{green!40!black},
  commentstyle=\itshape\color{purple!40!black},
  identifierstyle=\color{blue},
  stringstyle=\color{orange},
}

\lstdefinestyle{custompy}{
  belowcaptionskip=1\baselineskip,
  breaklines=true,
  numbers=none,
  frame=l,
  xleftmargin=\parindent,
  language=python,
  showstringspaces=false,
  basicstyle=\footnotesize\ttfamily,
  keywordstyle=\bfseries\color{green!40!black},
  commentstyle=\itshape\color{purple!40!black},
  identifierstyle=\color{blue},
  stringstyle=\color{orange},
}

\lstdefinestyle{customasm}{
  belowcaptionskip=1\baselineskip,
  frame=L,
  xleftmargin=\parindent,
  language=[x86masm]Assembler,
  basicstyle=\footnotesize\ttfamily,
  commentstyle=\itshape\color{purple!40!black},
}

\lstset{escapechar=@,style=custompy}
\renewcommand{\lstlistingname}{プログラム}

\begin{comment}
\makeatletter
\def\ps@plainfoot{%
  \let\@mkboth\@gobbletwo
  \let\@oddhead\@empty
  \def\@oddfoot{\normalfont\hfil-- \thepage\ --\hfil}%
  \let\@evenhead\@empty
  \let\@evenfoot\@oddfoot}
  \let\ps@plain\ps@plainfoot
  \renewcommand{\chapter}{%
  \if@openright\cleardoublepage\else\clearpage\fi
  \global\@topnum\z@
  \secdef\@chapter\@schapter}
\makeatother
\end{comment}

%
\newcommand{\maru}[1]{{\ooalign{%
\hfil\hbox{$\bigcirc$}\hfil\crcr%
\hfil\hbox{#1}\hfil}}}
%
\setlength{\textwidth}{\fullwidth}
\setlength{\textheight}{40\baselineskip}
\addtolength{\textheight}{\topskip}
\setlength{\voffset}{-0.55in}
%
\makeatletter
\newenvironment{tablehere}
  {\def\@captype{table}}
  {}
\newenvironment{figurehere}
  {\def\@captype{figure}}
  {}
\makeatother
%
\title{数独}
\author{smat1957@gmail.com\thanks{https://altema.is.tohoku.ac.jp/QA4U3/}}
\date{\today}
%
\begin{document}

\maketitle

\chapter{数独を量子アニーリングで解く}

数独は$M\times M$のブロックを、行方向に$M$列分、列方向に$M$行分並べた、全$M\times M=M^2$ブロック、従って$M^2\times M^2$個のセルからなる盤面で、どの行および列についても、また$M\times M$の各ブロックの中においても、同じ数値が2個以上現れてはならないという制約の下、各セルに$1\sim M^2$までの数値を一つずつ入れて盤面を埋めていくクイズ。

\section{問題の構成}

決定変数$q$を各セル毎に$M\times M=M^2$個用意する。$q_{i,j,n}$は、$i$行$j$列目のセル内の$M\times M=M^2$個の決定変数。\\ここで、$i, j, n\;\in\{1, 2, \cdots, M\times M=M^2\}$。

下の表は$3\times3$のブロック1個の例を表している。このブロックが横に3行、縦に3列、全部で9ブロックが並んでいる盤面がよく知られた数独問題になる。

\begin{center}
\begin{tabular}{|c||c|c| c |c||c|c| c |c||c|c| c |c|} \hline
  &\multicolumn{4}{|c||}{1列目($j=1$)}&\multicolumn{4}{|c||}{2列目($j=2$)}&\multicolumn{4}{|c|}{3列目($j=3$)}\\\hline\hline
  \multirow{2}{*}{1行目}&\multicolumn{4}{|c||}{セル($i=1, j=1$)}&\multicolumn{4}{|c||}{セル($i=1, j=2$)}&\multicolumn{4}{|c|}{セル($i=1, j=3$)}\\\cline{2-13}
  &1&2& $\cdots$ &9& 1&2& $\cdots$ &9& 1&2&$\cdots$ &9 \\\cline{2-13}
  ($i=1$)&$q_{111}$&$q_{112}$& $\cdots$ &$q_{119}$& $q_{121}$&$q_{122}$&$\cdots$&$q_{129}$& $q_{131}$&$q_{132}$&$\cdots$&$q_{139}$ \\\hline\hline
  %& & & $\cdots$ & & &  & &$\cdots$& &  & & & $\cdots$ & & \\\hline\hline
  \multirow{2}{*}{2行目}&\multicolumn{4}{|c||}{セル($i=2, j=1$)}&\multicolumn{4}{|c||}{セル($i=2, j=2$)}&\multicolumn{4}{|c|}{セル($i=2, j=3$)}\\\cline{2-13}
  &1&2& $\cdots$ &9& 1&2& $\cdots$ &9& 1&2&$\cdots$ &9 \\\cline{2-13}
  ($i=2$)&$q_{211}$&$q_{212}$& $\cdots$ &$q_{219}$& $q_{221}$&$q_{222}$&$\cdots$&$q_{229}$& $q_{231}$&$q_{232}$&$\cdots$&$q_{239}$ \\\hline\hline
  %& & & $\cdots$ & & &  & &$\cdots$& &  & & & $\cdots$ & & \\\hline\hline
  \multirow{2}{*}{3行目}&\multicolumn{4}{|c||}{セル($i=3, j=1$)}&\multicolumn{4}{|c||}{セル($i=3, j=2$)}&\multicolumn{4}{|c|}{セル($i=3, j=3$)}\\\cline{2-13}
  &1&2& $\cdots$ &9& 1&2& $\cdots$ &9& 1&2&$\cdots$ &9 \\\cline{2-13}
  ($i=3$)&$q_{311}$&$q_{312}$& $\cdots$ &$q_{319}$& $q_{321}$&$q_{322}$&$\cdots$&$q_{329}$& $q_{331}$&$q_{332}$&$\cdots$&$q_{339}$ \\\hline
  %& & & $\cdots$  & & &  & &$\cdots$& &  & & & $\cdots$ & & \\\hline
\end{tabular}
\end{center}

この決定変数は0か1かの2値変数で、数値$1\sim N$をそのセルに置く(1)か置かない(0)かを表している。

\newpage

制約条件は、次の様に考えることができる。($M\times M=M^2$を$N$と書く事にする)

\begin{enumerate}
\item 各セルの中では$q_1\sim q_N$の内でどれか1つだけが1になる(セルに2つ以上の数値は入らない)
\begin{equation*}
  f_1 = \sum_i^N\sum_j^N\bigg(\sum_n^N q_{i,j,n} - 1\bigg)^2
\end{equation*}
\item 同一の行(列)にあるセルの数値と同じ数値は、同じ行(列)の他のセルには入らない(第1項が行、第2項が列)
\begin{equation*}
  f_2 = \sum_i^N\sum_n^N\bigg(\sum_j^N q_{i,j,n} - 1\bigg)^2 + \sum_j^N\sum_n^N\bigg(\sum_i^N q_{i,j,n} - 1\bigg)^2
\end{equation*}
\item いずれのブロック($M\times M$)においても、その中のセルの数値は重複しない
\begin{eqnarray*}
  f_3 = \sum_{ブロック先頭セルのi_0,j_0}\sum_n^N\bigg(\sum_x^M\sum_y^M q_{i_0+x-1,j_0+y-1,n} - 1\bigg)^2
\end{eqnarray*}
\item いずれの行(列)方向の数値の和も同じ値S($=1+2+\cdots +N$)になる(第1項が行、第2項が列)
\begin{equation*}
  f_4 = \sum_i^N\bigg(\sum_j^N\sum_n^N n \cdot q_{i,j,n} - S\bigg)^2 + \sum_j^N\bigg(\sum_i^N\sum_n^N n \cdot q_{i,j,n} - S\bigg)^2
\end{equation*}
\item いずれのブロック($M\times M$)においても、その中のセルの数値の和は同じ値S($=1+2+\cdots +N$)になる
\begin{eqnarray*}
  f_5 = \sum_{ブロック先頭セルのi_0,j_0}\bigg(\sum_x^M\sum_y^M\sum_n^N n \cdot q_{i_0+x-1,j_0+y-1,n} - S\bigg)^2
\end{eqnarray*}
\item 予め数値$X\in\{1,\cdots,N\}$が決められている$I$行$J$列目のセルがある
\begin{eqnarray*}
  f_6 = \sum_{既定のセルI,J}\bigg(\sum_n^N n \cdot q_{I,J,n} - X\bigg)^2
\end{eqnarray*}
\end{enumerate}

\subsection{式の展開}

\begin{eqnarray*}
  f_1 &=& \sum_i^N\sum_j^N\bigg(\sum_n^N q_{i,j,n} - 1\bigg)^2\\
&=& \sum_i^N\sum_j^N\bigg(\sum_{n_1}^N\sum_{n_2}^N q_{i,j,n_1}q_{i,j,n_2} -2\sum_n^N q_{i,j,n} \bigg)\\
f_2 &=& \sum_i^N\sum_n^N\bigg(\sum_j^N q_{i,j,n} - 1\bigg)^2 + \sum_j^N\sum_n^N\bigg(\sum_i^N q_{i,j,n} - 1\bigg)^2\\
&=& \sum_i^N\sum_n^N\bigg(\sum_{j_1}^Nq_{i,j_1,n}\sum_{j_2}^Nq_{i,j_2,n} - 2\sum_j^N q_{i,j,n}\bigg)
 + \sum_j^N\sum_n^N\bigg(\sum_{i_1}^Nq_{i_1,j,n}\sum_{i_2}^Nq_{i_2,j,n} - 2\sum_i^N q_{i,j,n}\bigg)\\
 f_3 &=& \sum_{ブロック先頭セルのi_0,j_0}\sum_n^N\bigg(\sum_x^M\sum_y^M q_{i_0+x-1,j_0+y-1,n} - 1\bigg)^2\\
 &=& \sum_{ブロック先頭セルのi_0,j_0}\sum_n^N\bigg(\sum_{x_1}^M\sum_{y_1}^M q_{i_0+x_1-1,j_0
 +y_1-1,n}\sum_{x_2}^M\sum_{y_2}^M q_{i_0+x_2-1,j_0
 +y_2-1,n}\\
  &-& 2\sum_x^M\sum_y^M q_{i_0+x-1,j_0+y-1,n}\bigg)\\
 f_4 &=& \sum_i^N\bigg(\sum_j^N\sum_n^N n \cdot q_{i,j,n} - S\bigg)^2 + \sum_j^N\bigg(\sum_i^N\sum_n^N n \cdot q_{i,j,n} - S\bigg)^2\\
 &=& \sum_i^N\bigg(\sum_{j_1}^N\sum_{n_1}^N n_1 q_{i,j_1,n_1}\sum_{j_2}^N\sum_{n_2}^N n_2 q_{i,j_2,n_2} -2S\sum_j^N\sum_n^N n q_{i,j,n}\bigg)\\
 &+& \sum_j^N\bigg(\sum_{i_1}^N\sum_{n_1}^N n_1 q_{i_1,j,n_1}\sum_{i_2}^N\sum_{n_2}^N n_2 q_{i_2,j,n_2} -2S\sum_i^N\sum_n^N n q_{i,j,n}\bigg)\\
 f_5 &=& \sum_{ブロック先頭セルのi_0,j_0}\bigg(\sum_x^M\sum_y^M\sum_n^N n \cdot q_{i_0+x-1,j_0+y-1,n} - S\bigg)^2\\
 &=& \sum_{ブロック先頭セルのi_0,j_0}\bigg(\sum_{x_1}^M\sum_{y_1}^M\sum_{n_1}^N n_1 q_{i_0+x_1-1,j_0+y_1-1,n_1}\sum_{x_2}^M\sum_{y_2}^M\sum_{n_2}^N n_2 q_{i_0+x_2-1,j_0+y_2-1,n_2}\\
  &-& 2S\sum_x^M\sum_y^M\sum_n^N n q_{i_0+x-1,j_0+y-1}\bigg)\\
 f_6 &=& \sum_{既定のセルI,J,X}\bigg(X \cdot q_{I,J,n} - X\bigg)^2\\
 &=& \sum_{既定セルのI,J,X}\bigg( X q_{I,J,X} X q_{I,J,X} - 2 X Xq_{I,J,n}\bigg)\\
 &=& \sum_{既定セルのI,J,X}\bigg(q_{I,J,X}\cdot q_{I,J,X} - 2 q_{I,J,X}\bigg)\\
 &=& \sum_{既定セルのI,J,X}\bigg(-q_{I,J,X}\bigg)
\end{eqnarray*}

\section{式の展開と実装}

\begin{itemize}
\item 式を展開する上で留意する点は次の2点だけ
  \begin{itemize}
  \item[(1)] $0$か$1$の何れかの値しかとらない二値変数の場合$q^2=q$が成り立つ
  \item[(2)] 定数は最小化に関係ないので無視できる
  \end{itemize}
\item 展開した制約式に現れる$\sum$を、そのままfor文の繰り返しに移せばQUBOを生成できる
\item QUBOができたら、それを量子コンピュータのシミュレータである、SASampler()あるいはSQASampler()の第1引数に渡してあげると、計算結果のsamplesetを受け取ることができる
\item 数式上で$N$個の数値を$\sum_{i=1}^N$の様に扱っていても、プログラム上の始まりの値は$0$なので、全部で$N$個の数値をfor文で繰り返すとなると、終わりの値は$N-1$になる
\item また、盤面に置く数値は$0\sim N$の$N+1$個ではなくて、$1\sim N$の$N$個である事にも注意してプログラムする必要がある
\end{itemize}

\subsection{class}

\begin{lstlisting}[language=Python]
from openjij import SASampler, SQASampler
from collections import defaultdict, Counter
import numpy as np
\end{lstlisting}

\begin{lstlisting}[language=Python]
class NumberPlace:
  def __init__(self, M=2, FileN='data.txt'):
      self.M = M
      self.N = M * M
      S = 0
      for i in range(1, self.N+1):
          S += i
      self.S = S
      with open(FileN, 'r') as f:
          self.required = f.read().splitlines()
      self.idx = {}
      k = 0
      for i in range(self.N):
          for j in range(self.N):
              for n in range(self.N):
                  self.idx[(i,j,n)] = k
                  k += 1
      samplers = [SASampler(), SQASampler()]
      self.sampler = samplers[0]
      self.debug = False

  def get_param(self):
      return self.N, self.M, self.S, self.idx

  def block_ij(self):
      i0j0 = []
      for i in range(self.M):
          i0j0.append(self.M*i)
      return i0j0

  def belongs(self, I, J):
      i0j0 = self.block_ij()
      I0 = 0
      J0 = 0
      for n in i0j0:
          if n<=I<(n+3):
              I0 = n
          if n<=J<(n+3):
              J0 = n
      return I0, J0
\end{lstlisting}

\begin{verbatim}
  ブロック先頭セルの$i_0, j_0$は、
  $M=2$なら [(0,0),(0,2)],
            [(2,0),(2,2)]
  $M=3$なら [(0,0),(0,3),(0,6)],
            [(3,0),(3,3),(3,6)],
            [(6,0),(6,3),(6,6)]
  $M=2$なら [(M*0,M*0),(M*0,M*1)],
            [(M*1,M*0),(M*1,M*1)]
  $M=3$なら [(M*0,M*0),(M*0,M*1),(M*0,M*2)],
            [(M*1,M*0),(M*1,M*1),(M*1,M*2)],
            [(M*2,M*0),(M*2,M*1),(M*2,M*2)]
\end{verbatim}

\subsection{制約:$f_1$}

各セルの中では$q_1\sim q_N$の内でどれか1つだけが1になる(セルに2つ以上の数値は入らない)

\begin{eqnarray*}
  f_1 &=& \sum_i^N\sum_j^N\bigg(\sum_n^N q_{i,j,n} - 1\bigg)^2\\
  &=& \sum_i^N\sum_j^N\bigg(\sum_{n_1}^N\sum_{n_2}^N q_{i,j,n_1}q_{i,j,n_2} -2\sum_n^N q_{i,j,n} \bigg)
\end{eqnarray*}

\begin{lstlisting}[language=Python]
  def sub1(self, i, j, L, Q):
      N, _, _, idx = self.get_param()
      for n1 in range(N):
          Q[(idx[(i, j, n1)], idx[(i, j, n1)])] -= 2.0 * L
          for n2 in range(N):
              Q[(idx[(i, j, n1)], idx[(i, j, n2)])] += 1.0 * L

  def f1(self, L, Q):
      N, _, _, _ = self.get_param()
      for i in range(N):
          for j in range(N):
              self.sub1(i, j, L, Q)
      return Q
\end{lstlisting}

\subsection{制約:$f_2$}

同一の行(列)にあるセルの数値と同じ数値は、同じ行(列)の他のセルには入らない(第1項が行、第2項が列)

\begin{eqnarray*}
  f_2 &=& \sum_i^N\sum_n^N\bigg(\sum_j^N q_{i,j,n} - 1\bigg)^2 + \sum_j^N\sum_n^N\bigg(\sum_i^N q_{i,j,n} - 1\bigg)^2\\
  &=& \sum_i^N\sum_n^N\bigg(\sum_{j_1}^Nq_{i,j_1,n}\sum_{j_2}^Nq_{i,j_2,n} - 2\sum_j^N q_{i,j,n}\bigg)
  + \sum_j^N\sum_n^N\bigg(\sum_{i_1}^Nq_{i_1,j,n}\sum_{i_2}^Nq_{i_2,j,n} - 2\sum_i^N q_{i,j,n}\bigg)
\end{eqnarray*}

\begin{lstlisting}[language=Python]
  def sub2R(self, i, n, L, Q):
      N, _, _, idx = self.get_param()
      for j1 in range(N):
          Q[(idx[(i, j1, n)], idx[(i, j1, n)])] -= 2.0 * L
          for j2 in range(N):
              Q[(idx[(i, j1, n)], idx[(i, j2, n)])] += 1.0 * L

  def sub2C(self, j, n, L, Q):
    N, _, _, idx = self.get_param()
      for i1 in range(N):
          Q[(idx[(i1, j, n)], idx[(i1, j, n)])] -= 2.0 * L
          for i2 in range(N):
              Q[(idx[(i1, j, n)], idx[(i2, j, n)])] += 1.0 * L

  def f2(self, L, Q):
      N, _, _, _ = self.get_param()
      for i in range(N):
          for n in range(N):
              self.sub2R(i, n, L, Q)
      for j in range(N):
          for n in range(N):
              self.sub2C(j, n, L, Q)
      return Q
\end{lstlisting}

\subsection{制約:$f_3$}

いずれのブロック($M\times M$)においても、その中のセルの数値は重複しない

\begin{eqnarray*}
  f_3 &=& \sum_{ブロック先頭セルのi_0,j_0}\sum_n^N\bigg(\sum_x^M\sum_y^M q_{i_0+x-1,j_0+y-1,n} - 1\bigg)^2\\
  &=& \sum_{ブロック先頭セルのi_0,j_0}\sum_n^N\bigg(\sum_{x_1}^M\sum_{y_1}^M q_{i_0+x_1-1,j_0
  +y_1-1,n}\sum_{x_2}^M\sum_{y_2}^M q_{i_0+x_2-1,j_0
  +y_2-1,n}\\
   &-& 2\sum_x^M\sum_y^M q_{i_0+x-1,j_0+y-1,n}\bigg)
\end{eqnarray*}

$i_0, j_0, n, x_1, x_2, y_1, y_2$は、どれも$0$から始まるインデックス。
QUBOのインデックスも$0$始まりなので、数式上$i_0+x-1, j_0+y-1, n$などと表現している部分は、プログラム上で$i_0+x, j_0+y,n$となる。

\begin{lstlisting}[language=Python]
  def sub3(self, i0, j0, n, L, Q):
      N, M, _, idx = self.get_param()
      for x1 in range(M):
          for y1 in range(M):
              #if self.debug: print(f'debug: i0+x1={i0+x1}, j0+y1={j0+y1}, n={n}')
              Q[(idx[(i0+x1,j0+y1,n)],idx[(i0+x1,j0+y1,n)])] -= 2.0 * L
              for x2 in range(M):
                  for y2 in range(M):
                      Q[(idx[(i0+x1,j0 +y1,n)], idx[(i0+x2,j0+y2,n)])] += 1.0 * L

  def f3(self, L, Q):
      N, _, _, idx = self.get_param()
      i0j0 = self.block_ij()
      for i0 in i0j0:
          for j0 in i0j0:
              for n in range(N):
                  #if self.debug: print(f'debug:(i0,j0)=({i0},{j0})')
                  self.sub3(i0, j0, n, L, Q)
      return Q
\end{lstlisting}

\subsection{制約:$f_4$}

いずれの行(列)方向の数値の和も同じ値S($=1+2+\cdots +N$)になる(第1項が行、第2項が列)

\begin{eqnarray*}
  f_4 &=& \sum_i^N\bigg(\sum_j^N\sum_n^N n \cdot q_{i,j,n} - S\bigg)^2 + \sum_j^N\bigg(\sum_i^N\sum_n^N n \cdot q_{i,j,n} - S\bigg)^2\\
  &=& \sum_i^N\bigg(\sum_{j_1}^N\sum_{n_1}^N n_1 q_{i,j_1,n_1}\sum_{j_2}^N\sum_{n_2}^N n_2 q_{i,j_2,n_2} -2S\sum_j^N\sum_n^N n q_{i,j,n}\bigg)\\
  &+& \sum_j^N\bigg(\sum_{i_1}^N\sum_{n_1}^N n_1 q_{i_1,j,n_1}\sum_{i_2}^N\sum_{n_2}^N n_2 q_{i_2,j,n_2} -2S\sum_i^N\sum_n^N n q_{i,j,n}\bigg)
\end{eqnarray*}

$i, j, n$は、どれも$0$から始まるインデックス。
また、QUBOのインデックスも$0$始まり。
したがって、数式上の$i, j, n$などのインデックスは、そのままプログラム上でも$i, j, n$となる。
一方インデックスではなくて、盤面に配置するオブジェクト$1\sim N$を表している$n$は、数式上での$\sum_n$の$1$始まりが、当該部分に対応するプログラム上で$n$は$0$始まりの繰り返し文になるので$(n+1)$を掛けるようにしている。

\begin{lstlisting}[language=Python]
  def sub4R(self, i, L, Q):
      N, _, S, idx = self.get_param()
      for j1 in range(N):
          for n1 in range(N):
              Q[(idx[(i,j1,n1)],idx[(i,j1,n1)])] -= 2.0 * (n1+1) * S * L
              for j2 in range(N):
                  for n2 in range(N):
                      Q[(idx[(i,j1,n1)],idx[(i,j2,n2)])] += (n1+1) * (n2+1) * L

  def sub4C(self, j, L, Q):
      N, _, S, idx = self.get_param()
      for i1 in range(N):
          for n1 in range(N):
              Q[(idx[(i1,j,n1)],idx[(i1,j,n1)])] -= 2.0 * (n1+1) * S * L
              for i2 in range(N):
                  for n2 in range(N):
                      Q[(idx[(i1,j,n1)],idx[(i2,j,n2)])] += (n1+1) * (n2+1) * L

  def f4(self, L, Q):
      N, _, _, _ = self.get_param()
      for i in range(N):
          self.sub4R(i, L, Q)
      for j in range(N):
          self.sub4C(j, L, Q)
      return Q
\end{lstlisting}

\subsection{制約:$f_5$}

いずれのブロック($M\times M$)においても、その中のセルの数値の和は同じ値S($=1+2+\cdots +N$)になる

\begin{eqnarray*}
  f_5 &=& \sum_{ブロック先頭セルのi_0,j_0}\bigg(\sum_x^M\sum_y^M\sum_n^N n \cdot q_{i_0+x-1,j_0+y-1,n} - S\bigg)^2\\
  &=& \sum_{ブロック先頭セルのi_0,j_0}\bigg(\sum_{x_1}^M\sum_{y_1}^M\sum_{n_1}^N n_1 q_{i_0+x_1-1,j_0+y_1-1,n_1}\sum_{x_2}^M\sum_{y_2}^M\sum_{n_2}^N n_2 q_{i_0+x_2-1,j_0+y_2-1,n_2}\\
   &-& 2S\sum_x^M\sum_y^M\sum_n^N n q_{i_0+x-1,j_0+y-1,n}\bigg)
\end{eqnarray*}

\begin{lstlisting}[language=Python]
  def sub5(self, i0, j0, L, Q):
      N, M, S, idx = self.get_param()
      for x1 in range(M):
          for y1 in range(M):
              for n1 in range(N):
                  Q[(idx[(i0+x1,j0+y1,n1)],idx[(i0+x1,j0+y1,n1)])] -= 2.0 * (n1+1) * S * L
                  for x2 in range(M):
                      for y2 in range(M):
                          for n2 in range(N):
                              Q[(idx[(i0+x1,j0+y1,n1)],idx[(i0+x2,j0+y2,n2)])] += (n1+1) * (n2+1) * L

    def f5(self, L, Q):
      i0j0 = self.block_ij()
      for i0 in i0j0:
          for j0 in i0j0:
              self.sub5(i0, j0, L, Q)
      return Q
\end{lstlisting}

\subsection{制約:$f_6$}

予め数値$X\in\{1,\cdots,N\}$が決められている$I$行$J$列目のセルがある
\begin{eqnarray*}
  f_6 &=& \sum_{既定のI, J, X}\bigg(q_{I, J, X} - 1\bigg)^2\\
  &=& \sum_{既定セルのI,J,X}\bigg( q_{I,J,X}\cdot q_{I,J,X} - 2\cdot q_{I,J,X}\bigg)\\
  &=& \sum_{既定のI,J,X}\bigg(q_{I,J,X}-2q_{I,J,X}\bigg)\\
  &=& \sum_{既定のI,J,X}\bigg(-q_{I,J,X}\bigg)
\end{eqnarray*}

\begin{lstlisting}[language=Python]
  def f6(self, I, J, X, L, Q):
      _, _, _, idx = self.get_param()
      Q[(idx[(I, J, X-1)], idx[(I, J, X-1)])] -= L
      return Q
\end{lstlisting}
あるいは、より厳しくして探索空間を小さくすることを期待して、
\begin{eqnarray*}
  f_{60} &=& \sum_{既定のI,J,X}\bigg(\sum_n^N n q_{I,J,n} - X\bigg)^2=\sum_{既定のI,J,X}\bigg(\sum_{n_1}^N\sum_{n_2}^N n_1 n_2 q_{I,J,n_1}q_{I,J,n_2}- 2 X\sum_n^N n q_{I,J,n}\bigg)\\
  f_{61} &=& \sum_{既定のI}\bigg(\sum_j^N X q_{I,j,X} - X\bigg)^2=\sum_{既定のI}X^2\bigg(\sum_{j_1}^N\sum_{j_2}^N q_{I,j_1,X}q_{I,j_2,X}- 2\sum_j^N q_{I,j,X}\bigg)\\
  f_{62} &=& \sum_{既定のJ}\bigg(\sum_i^N X q_{i,J,X} - X\bigg)^2=\sum_{既定のJ}X^2\bigg(\sum_{i_1}^N\sum_{i_2}^N q_{i_1,J,X}q_{i_2,J,X}- 2\sum_i^N q_{i,J,X}\bigg)\\
  f_{63} &=& \sum_{既定のI,Jセルが属するブロックの先頭I_0,J_0}\bigg(\sum_x^M\sum_y^M X q_{I_0+x-1,J_0+y-1,X} -X\bigg)^2\\
  &=& \sum_{既定のI,Jセルが属するブロックの先頭I_0,J_0}X^2\bigg(\sum_{x_1}^M\sum_{y_1}^M q_{I_0
  +..,j_0+..,X}\sum_{x_2}^M\sum_{y_2}^M q_{I_0+..,J_0
  +..,X}-2\sum_x^M\sum_y^M q_{I_0+..,J_0+..,X}\bigg)
\end{eqnarray*}

ここで、$f_{6another} = f_{60} + f_{61} + f_{62} +f_{63}$とすることも考えられる。

\begin{lstlisting}[language=Python]
  def f60(self, I, J, X, L, Q):
      N, _, _, idx = self.get_param()
      for n1 in range(N):
          Q[(idx[(I,J,n1)],idx[(I,J,n1)])] -= 2.0 * X * (n1 + 1) * L
          for n2 in range(N):
              Q[(idx[(I,J,n1)],idx[(I,J,n2)])] += (n1+1) * (n2+1) * L

  def f61(self, I, J, X, L, Q):
      N, _, _, idx = self.get_param()
      for j1 in range(N):
          Q[(idx[(I,j1,X-1)],idx[(I,j1,X-1)])] -= 2.0 * L
          for j2 in range(N):
              Q[(idx[(I,j1,X-1)],idx[(I,j2,X-1)])] += 1.0 * L

  def f62(self, I, J, X, L, Q):
      N, _, _, idx = self.get_param()
      for i1 in range(N):
          Q[(idx[(i1,J,X-1)],idx[(i1,J,X-1)])] -= 2.0 * L
          for i2 in range(N):
              Q[(idx[(i1,J,X-1)],idx[(i2,J,X-1)])] += 1.0 * L

  def f63(self, I, J, X, L, Q):
      N, M, _, idx = self.get_param()
      I0, J0 = self.belongs(I, J)
      for x1 in range(M):
          for y1 in range(M):
              Q[(idx[(I0+x1,J0+y1,X-1)],idx[(I0+x1,J0+y1,X-1)])] -= 2.0 * L
              for x2 in range(M):
                  for y2 in range(M):
                      Q[(idx[(I0+x1,J0+y1,X-1)],idx[(I0+x2,J0+y2,X-1)])] += 1.0 * L

  def f6_another(self, I, J, X, L, Q):
      self.f60(I, J, X, L, Q)
      self.f61(I, J, X, L, Q)
      self.f62(I, J, X, L, Q)
      self.f63(I, J, X, L, Q)
      return Q
\end{lstlisting}

しかしながら、$f_{61}, f_{62, f_{63}}$は、行、列、ブロックのそれぞれの中では、数値の重複はないとする制約2及び制約3と同じであり、また、$f_{60}$についても、各セルの中で数値は1つだとする制約1と先の$f_6$の両者を指定しているなら、同じであると考えられることから、先に指定した$f_6$だけで十分だと考える。

\subsection{評価関数:$f$}

\begin{eqnarray*}
  f = \lambda_1\cdot f_1 + \lambda_2\cdot(f_2 + f_3) + \lambda_3\cdot(f_4 + f_5) + \lambda_4\cdot (\sum_{既定}f_6)
\end{eqnarray*}

制約の$f_1, f_2, f_3$が満たされると、制約$f_4, f_5$は自動的に満たされるので、制約$f_4, f_5$は$0<\lambda_3$の時だけの選択として、除外もできる様にした。

\begin{lstlisting}[language=Python]
    def f(self, lagrange1=1.0, lagrange2=1.0, lagrange3=1.0, lagrange4=1.0):
      Q = defaultdict(lambda: 0)
      _ = self.f1(lagrange1, Q)
      _ = self.f2(lagrange2, Q)
      _ = self.f3(lagrange2, Q)
      if 0.0 < lagrange3:
          _ = self.f4(lagrange3, Q)
          _ = self.f5(lagrange3, Q)
      for a in self.required:
          IJX = a.split(',')
          _ = self.f6(int(IJX[0]), int(IJX[1]), int(IJX[2]), lagrange4, Q)
      return Q

  def solv(self, Q, num_reads=1):
      sampleset = self.sampler.sample_qubo(Q, num_reads=num_reads)
      return sampleset

  def result(self, sampleset):
      N, _, _, idx = self.get_param()
      result = [i for i in sampleset.first[0].values()]
      ans = [[None] * N for _ in range(N)]
      for i in range(N):
          for j in range(N):
              for n in range(N):
                  if result[idx[(i,j,n)]] == 1:
                      ans[i][j] = n+1
      return ans
\end{lstlisting}

\subsubsection{出力結果のチェック}

出力された結果を、ふるいにかける仕掛け

\begin{lstlisting}[language=Python]
    def evaluate(self, sampleset):
      # Extract sample solutions, energies, and sort them by frequency
      samples = sampleset.record['sample']
      energies = sampleset.record['energy']
      # Combine solutions and corresponding energies
      sample_data = [(tuple(sample), energy) for sample, energy in zip(samples, energies)]
      # Sort the results by appearance frequency and then energy
      sample_frequency = Counter(sample for sample, _ in sample_data)
      # Print sorted results by frequency and include energy
      if self.debug:
          print("\nSorted samples by frequency and energy:")
          for solution, freq in sample_frequency.most_common():
              energy = next(energy for sample, energy in sample_data if sample == solution)
              print(f"Sample: {solution}, Frequency: {freq}, Energy: {energy:+.2f}")
      return sample_data, sample_frequency

  def check1(self, a):
      N, M, _, _ = self.get_param()
      b = np.array(a).reshape(N*N, N)

      # 既定値は正しい?
      for a in self.required:
          IJX = a.split(',')
          i = int(IJX[0])*N + int(IJX[1])
          n = int(IJX[2]) - 1
          if b[i][n]!=1:
              if self.debug: print(f'!: 既定値が違う:({IJX[0]},{IJX[1]}){IJX[2]}!={b[i]}')
              return False

      # 各セルに数値は1つ?
      for i in range(N*N):
          s = 0
          for n in range(N):
              s += b[i][n]
          if s != 1:
              if self.debug: print(f'!: セルの中の数値が1つでない{i:3d}:{b[i]}')
              return False

              # 各ブロックに重複する数値はない?
      i0j0 = self.block_ij()
      for i0 in i0j0:
          for j0 in i0j0:
              for n in range(N):
                  ary = []
                  s = 0
                  for x in range(M):
                      for y in range(M):
                          bidx = (i0+x)*N + j0+y
                          s += b[bidx][n]
                          ary.append(b[bidx][n])
                  if s != 1:
                      if self.debug: print(f'!: ブロック内で数値が重複:{np.array(ary)}')
                      return False
      #
      for n in range(N):
          # 各行に重複する数値はない?
          for i in range(N):
              ary = []
              s = 0
              for j in range(N):
                  bidx = i * N + j
                  s += b[bidx][n]
                  ary.append(b[bidx][n])
              if s != 1:
                  if self.debug: print(f'!: 行で数値が重複:{np.array(ary)}')
                  return False
          # 各列に重複する数値はない?
          for j in range(N):
              ary = []
              s = 0
              for i in range(N):
                  bidx = i * N + j
                  s += b[bidx][n]
                  ary.append(b[bidx][n])
              if s != 1:
                  if self.debug: print(f'!: 列で数値が重複:{np.array(ary)}')
                  return False
      #
      return True

  def check2(self, a):
      N, M, S, _ = self.get_param()
      b = np.array(a).reshape(N, N)
      '''
      # 既定値は正しい?
      for a in self.required:
          IJX = a.split(',')
          if b[int(IJX[0])][int(IJX[1])]!=int(IJX[2]):
              if self.debug: print(f'!: 既定値が違う:({IJX[0]},{IJX[1]}){IJX[2]}!={b[int(IJX[0])][int(IJX[1])]}')
              return False
      '''
      # 各行の数値の和はS?
      for i in range(N):
          s = 0
          for j in range(N):
              s += b[i][j]
          if s != S:
              if self.debug: print(f'!: 行の総和={s}!={S}')
              return False
      # 各列の数値の和はS?
      for j in range(N):
          s = 0
          for i in range(N):
              s += b[i][j]
          if s != S:
              if self.debug: print(f'!: 列の総和={s}!={S}')
              return False
      # 各ブロックの数値の和はS?
      i0j0 = self.block_ij()
      for i in i0j0:
          for j in i0j0:
              s = 0
              for x in range(M):
                  for y in range(M):
                      #print(i+x,j+y)
                      s += b[i+x][j+y]
              if s != S:
                  if self.debug: print(f'!: ブロック内の総和={s}!={S}')
                  return False
      #
      return True

  def decode(self, a):
      N, M, _, _ = self.get_param()
      b = np.array(a).reshape(N**2, N)
      mat = []
      for v in b:
          num = 0
          for i, u in enumerate(v):
              if u==1:
                  num = i+1
          mat.append(num)
      return mat

  def print_shape(self):
      for i in range(self.N):
          print(f'{i}:', end='\t')
          for j in range(self.N):
              for a in self.required:
                  IJX = a.split(',')
                  if i==int(IJX[0]) and j==int(IJX[1]):
                      print(int(IJX[2]), end=' ')
                      break
              else:
                  print('_', end=' ')
          print()
\end{lstlisting}

\subsection{main}

\begin{lstlisting}[language=Python]
if __name__ == '__main__':
    KiteiF = 'dataA250429.txt'
    M = 3
    sudoku = NumberPlace(M, KiteiF)
    sudoku.print_shape()
    #lagrange1 = 40.0      # 数値に重複なし
    #lagrange2 =  5.4      # 行、列、ブロック、で重複なし
    #lagrange3 =  0.0      # 和はS
    #lagrange4 =  5.1      # 既定セル
    lagrange1 = 90.0                 # 数値に重複なし
    lagrange2 = lagrange1 * 0.8      # 行、列、ブロック、で重複なし
    lagrange3 = -0.0                 # 和はS
    lagrange4 = lagrange1 * 3.0      # 既定セル
    Q = sudoku.f(lagrange1, lagrange2, lagrange3, lagrange4)
    num_reads = 1000
    sampleset = sudoku.solv(Q, num_reads)
    ans = sudoku.result(sampleset)
    print(*ans, sep='\n')
    #
    sudoku.debug = True
    for sample in sampleset.record['sample']:
        if sudoku.check1(sample):
            #if sudoku.debug: print('check1 Passed!')
            a = sudoku.decode(sample)
            #if sudoku.check2(a):
                #if sudoku.debug: print('check2 Passed!')
            print(np.array(a).reshape(M*M, M*M))
            print()
            break
\end{lstlisting}

%\newpage

\section{実行結果}

\subsubsection{→ 4x4 の場合}

既定セルの値:data4.txt
\begin{verbatim}
  0, 1, 1
  1, 0, 2
  3, 0, 4
  3, 3, 1
\end{verbatim}

期待したのは次の状態
\begin{verbatim}
  [3, 1, 4, 2]
  [2, 4, 1, 3]
  [1, 2, 3, 4]
  [4, 3, 2, 1]
\end{verbatim}

実行結果は次の通り。惜しいが正解ではない
\begin{verbatim}
  0:	_ 1 _ _
  1:	2 _ _ _
  2:	_ _ _ _
  3:	4 _ _ 1

  [3, 1, 4, 2]
  [2, 4, 1, 4]
  [1, 2, 4, 3]
  [4, 3, 2, 1]
\end{verbatim}

何度かやっていると、辛うじて期待するものが出ることもある
\begin{verbatim}
  [[3 1 4 2]
   [2 4 1 3]
   [1 3 2 4]
   [4 2 3 1]]
\end{verbatim}

この時のパラメータは、
\begin{verbatim}
  #lagrange1 = 40.0      # 数値に重複なし
  #lagrange2 =  5.4      # 行、列、ブロック、で重複なし
  #lagrange3 =  0.0      # 和はS
  #lagrange4 =  5.1      # 既定セル
\end{verbatim}

(lagrange1 〜 lagrange4 の値のバランスがデリケートだ)

\subsubsection{→ 9x9 の場合}

dataA250429.txt が解けた
\begin{verbatim}
  0:	_ 6 _ 5 _ _ _ 4 _
  1:	4 _ _ _ 9 _ 8 _ _
  2:	3 _ 1 _ _ 6 _ 5 _
  3:	_ 3 _ 7 _ _ _ _ 8
  4:	_ _ 6 _ _ _ 9 _ _
  5:	5 _ _ _ _ 4 _ 1 _
  6:	_ 2 _ 4 _ _ 7 _ 3
  7:	_ _ 9 _ 3 _ _ _ 1
  8:	_ 8 _ _ _ 1 _ 2 _

  [[9 6 8 5 2 3 1 4 7]
   [4 5 2 1 9 7 8 3 6]
   [3 7 1 8 4 6 2 5 9]
   [2 3 4 7 1 9 5 6 8]
   [8 1 6 3 5 2 9 7 4]
   [5 9 7 6 8 4 3 1 2]
   [1 2 5 4 6 8 7 9 3]
   [7 4 9 2 3 5 6 8 1]
   [6 8 3 9 7 1 4 2 5]]
\end{verbatim}

この時のパラメータは、
\begin{verbatim}
  #lagrange1 = 90.0               # 数値に重複なし
  #lagrange2 = lagrange1 * 0.8    # 行、列、ブロック、で重複なし
  #lagrange3 = -0.0               # 和はS
  #lagrange4 = lagrange1 * 3.0    # 既定セル
\end{verbatim}

dataD250411.txt も解けた
\begin{verbatim}
  0:	8 _ _ 2 _ _ _ 7 _
  1:	_ _ 1 _ 3 _ 5 _ _
  2:	_ 9 _ _ _ 6 _ _ _
  3:	_ _ _ 5 _ _ _ 2 _
  4:	7 _ _ _ 8 _ _ _ 1
  5:	_ 6 _ _ _ 2 _ _ _
  6:	_ _ _ 7 _ _ _ 4 _
  7:	_ _ 3 _ 1 _ 8 _ _
  8:	_ 2 _ _ _ 9 _ _ 6

[[8 3 6 2 9 5 1 7 4]
 [2 4 1 8 3 7 5 6 9]
 [5 9 7 1 4 6 2 3 8]
 [3 8 9 5 6 1 4 2 7]
 [7 5 2 4 8 3 6 9 1]
 [1 6 4 9 7 2 3 8 5]
 [6 1 5 7 2 8 9 4 3]
 [9 7 3 6 1 4 8 5 2]
 [4 2 8 3 5 9 7 1 6]]
\end{verbatim}

この時のパラメータは、
\begin{verbatim}
  lagrange1 = 90.0              # 数値に重複なし
  lagrange2 = lagrange1 * 0.67  # 行、列、ブロック、で重複なし
  lagrange3 = -0.0              # 和はS
  lagrange4 = lagrange1 * 3     # 既定セル
\end{verbatim}


dataE250410.txt も解けた
\begin{verbatim}
0:	3 _ _ _ 4 _ 5 _ _
1:	_ _ 1 6 _ _ _ 7 _
2:	_ 6 _ _ _ _ _ _ 2
3:	_ _ _ 9 _ _ 3 _ _
4:	7 _ _ _ 5 _ _ _ 4
5:	_ _ 8 _ _ 1 _ _ _
6:	4 _ _ _ _ _ _ 1 _
7:	_ 3 _ _ _ 6 8 _ _
8:	_ _ 2 _ 7 _ _ _ 9

[[3 2 7 1 4 8 5 9 6]
 [9 5 1 6 2 3 4 7 8]
 [8 6 4 7 9 5 1 3 2]
 [2 4 5 9 6 7 3 8 1]
 [7 1 3 8 5 2 9 6 4]
 [6 9 8 4 3 1 7 2 5]
 [4 7 6 5 8 9 2 1 3]
 [5 3 9 2 1 6 8 4 7]
 [1 8 2 3 7 4 6 5 9]]
\end{verbatim}

この時のパラメータは、
\begin{verbatim}
  lagrange1 = 90.0              # 数値に重複なし
  lagrange2 = lagrange1 * 0.8   # 行、列、ブロック、で重複なし
  lagrange3 = -0.0              # 和はS
  lagrange4 = lagrange1 * 3.0   # 既定セル
\end{verbatim}

\section{プログラムの全体}

\lstinputlisting[caption=数独,label=p01]{NumberPlace.py}

\begin{thebibliography}{9}
  \bibitem{b1} 西森秀稔、大関真之, 量子アニーリングの基礎, 共立出版
  \bibitem{b2} 数独 - 技術リソース - Fixstars Amplify - 量子コンピューティング クラウド\\ \url{https://amplify.fixstars.com/ja/techresources/application/sudoku/}
  \bibitem{b3} D-waveの量子アニーリングのクラウドサービスLeapで数独を解いてみる\\
  \url{https://zenn.dev/airev/articles/airev-quantum-02}
  \bibitem{b4} 量子アニーリングを駆使して数独を解いてみた - IMACEL Academy -人工知能・画像解析の技術応用に向けて-| エルピクセル株式会社\\
\url{https://lp-tech.net/articles/jbkhW}
  \bibitem{b5} 21-19. 量子アニーリング(QUBO)でナンプレ(数独)と不等号ナンプレを解く | Vignette & Clarity(ビネット&クラリティ)\\
\url{https://vigne-cla.com/21-19/}
\bibitem{b6} 量子コンピュータで論理パズルを解いてみた #Python - Qiita\\
\url{https://qiita.com/tax_nalgo/items/48cf1ee47549fbf3b386}
\end{thebibliography}

\end{document}