% latex uft-8
\documentclass[uplatex,dvipdfmx,a4paper,11pt,oneside,openany]{jsbook}
%
\usepackage[dvipdfmx]{graphicx}
\usepackage{amsmath,amssymb}
\usepackage{bm}
\usepackage{physics}
\usepackage{graphicx}
\usepackage{ascmac}
\usepackage{amsmath}
\usepackage{setspace}
\usepackage{here}
\usepackage{fancybox}
\usepackage{url}
\usepackage{listings,jlisting} %日本語のコメントアウトをする場合jlistingが必要
\usepackage{xcolor}
\usepackage{comment}
\usepackage{multicol}
\usepackage{amsmath, amssymb}
\usepackage{type1cm}

\definecolor{mygreen}{rgb}{0,0.6,0}
\definecolor{mygray}{rgb}{0.5,0.5,0.5}
\definecolor{mymauve}{rgb}{0.58,0,0.82}

%\begin{comment}
\lstset{
  backgroundcolor=\color{white},   % choose the background color; you must add \usepackage{color} or \usepackage{xcolor}; should come as last argument
  basicstyle=\footnotesize,        % the size of the fonts that are used for the code
  breakatwhitespace=false,         % sets if automatic breaks should only happen at whitespace
  breaklines=true,                 % sets automatic line breaking
  captionpos=b,                    % sets the caption-position to bottom
  commentstyle=\color{mygreen},    % comment style
  deletekeywords={...},            % if you want to delete keywords from the given language
  escapeinside={\%*}{*)},          % if you want to add LaTeX within your code
  extendedchars=true,              % lets you use non-ASCII characters; for 8-bits encodings only, does not work with UTF-8
  firstnumber=1000,                % start line enumeration with line 1000
  frame=single,	                   % adds a frame around the code
  keepspaces=true,                 % keeps spaces in text, useful for keeping indentation of code (possibly needs columns=flexible)
  keywordstyle=\color{blue},       % keyword style
  language=Octave,                 % the language of the code
  morekeywords={*,...},            % if you want to add more keywords to the set
  numbers=left,                    % where to put the line-numbers; possible values are (none, left, right)
  numbersep=5pt,                   % how far the line-numbers are from the code
  numberstyle=\tiny\color{mygray}, % the style that is used for the line-numbers
  rulecolor=\color{black},         % if not set, the frame-color may be changed on line-breaks within not-black text (e.g. comments (green here))
  showspaces=false,                % show spaces everywhere adding particular underscores; it overrides 'showstringspaces'
  showstringspaces=false,          % underline spaces within strings only
  showtabs=false,                  % show tabs within strings adding particular underscores
  stepnumber=2,                    % the step between two line-numbers. If it's 1, each line will be numbered
  stringstyle=\color{mymauve},     % string literal style
  tabsize=2,	                   % sets default tabsize to 2 spaces
  title=\lstname                   % show the filename of files included with \lstinputlisting; also try caption instead of title
}
%\end{comment}

\lstdefinelanguage{mypy}{
  % リテラルと演算記号
  morekeywords=[1]{+=,=,==,!=,!,>,<,>=,<=,++,-,+,*,\%,/},
  % 予約語
  morekeywords=[2]{False,None,True,and,as,assert,async,await,break,
    class,continue,def,del,elif,else,except,finally,for,from,global,
    if,import,in,is,lambda,nonlocal,not,or,pass,raise,return,try,while,
    with,yield,print,match,case
  },
  % 識別子
  morekeywords=[3]{defined_func},
  % 区切り文字を強制的に色付け
  literate=*{.}{{\color{delimiter}.}}1
            {,}{{\color{delimiter},}}1 {:}{{\color{delimiter}:}}1
            {)}{{\color{delimiter})}}1 {(}{{\color{delimiter}(}}1
            {[}{{\color{delimiter}[}}1 {]}{{\color{delimiter}]}}1
            {\{}{{\color{delimiter}\{}}1 {\}}{{\color{delimiter}\}}}1,
  % 大文字小文字を区別
  sensitive=true,
  % 行コメントの設定
  morecomment=[l]{\#},
  % Stringリテラルの設定
  morestring=[b]{\'},
  morestring=[b]{\"},
  % 単語として扱う文字
  alsoletter={\%<>=+-*\/1234567890!},
  % 枠
  frame=none,
  % 長くなったら途中で改行
  breaklines=true,
  % 自動改行時のインデント
  breakindent=12pt,
  % 文字間隔を一定に
  columns=fixed,
  % 文字の横のサイズを小さく
  basewidth=0.5em,
  % 行番号を左に
  numbers=left,
  % 行番号の書式
  numberstyle={\scriptsize\color{white}},
  % 行番号の増加数は1=連番に
  stepnumber=1,
  % フレームの左の余白
  framexleftmargin=18pt,
  % スペースを省略せず保持
  keepspaces=true,
  % インデントサイズ
  tabsize=4,
  backgroundcolor={},                                   % 背景色=透明
  basicstyle={\small\ttfamily\color{white}},            % 通常部分の書式
  identifierstyle={\small\color{white}},                % 識別子の書式
  commentstyle={\small\color{comment}},                 % コメントの書式
  keywordstyle=[1]{\small\bfseries\color{literal}},     % リテラルと演算記号の書式
  keywordstyle=[2]{\small\bfseries\color{reserved}},    % 予約語の書式
  keywordstyle=[3]{\small\bfseries\color{identifier}},  % 自分で定義した識別子の書式
  stringstyle={\small\ttfamily\color{literal}},         % 文字列の書式
}
%ここからソースコードの表示に関する設定
\lstdefinestyle{customc}{
  belowcaptionskip=1\baselineskip,
  breaklines=true,
  numbers=left,
  frame=single,
  xleftmargin=\parindent,
  language=C,
  showstringspaces=false,
  basicstyle=\footnotesize\ttfamily,
  keywordstyle=\bfseries\color{green!40!black},
  commentstyle=\itshape\color{purple!40!black},
  identifierstyle=\color{blue},
  stringstyle=\color{orange},
}

\lstdefinestyle{custompy}{
  belowcaptionskip=1\baselineskip,
  breaklines=true,
  numbers=none,
  frame=l,
  xleftmargin=\parindent,
  language=python,
  showstringspaces=false,
  basicstyle=\footnotesize\ttfamily,
  keywordstyle=\bfseries\color{green!40!black},
  commentstyle=\itshape\color{purple!40!black},
  identifierstyle=\color{blue},
  stringstyle=\color{orange},
}

\lstdefinestyle{customasm}{
  belowcaptionskip=1\baselineskip,
  frame=L,
  xleftmargin=\parindent,
  language=[x86masm]Assembler,
  basicstyle=\footnotesize\ttfamily,
  commentstyle=\itshape\color{purple!40!black},
}

\lstset{escapechar=@,style=custompy}
\renewcommand{\lstlistingname}{プログラム}

\begin{comment}
\makeatletter
\def\ps@plainfoot{%
  \let\@mkboth\@gobbletwo
  \let\@oddhead\@empty
  \def\@oddfoot{\normalfont\hfil-- \thepage\ --\hfil}%
  \let\@evenhead\@empty
  \let\@evenfoot\@oddfoot}
  \let\ps@plain\ps@plainfoot
  \renewcommand{\chapter}{%
  \if@openright\cleardoublepage\else\clearpage\fi
  \global\@topnum\z@
  \secdef\@chapter\@schapter}
\makeatother
\end{comment}

%
\newcommand{\maru}[1]{{\ooalign{%
\hfil\hbox{$\bigcirc$}\hfil\crcr%
\hfil\hbox{#1}\hfil}}}
%
\setlength{\textwidth}{\fullwidth}
\setlength{\textheight}{40\baselineskip}
\addtolength{\textheight}{\topskip}
\setlength{\voffset}{-0.55in}
%
\makeatletter
\newenvironment{tablehere}
  {\def\@captype{table}}
  {}
\newenvironment{figurehere}
  {\def\@captype{figure}}
  {}
\makeatother

%
\begin{document}

\chapter{Eight Queen Problem}

\section{量子アニーリングでN-クィーン問題を解く}

\section{定式化}

N-Queen問題を考えてみる.
$x_{i,j}$は,$i$行$j$列目のマスを, 0にするか1にするかを表すものとする. 例えば、\\
2行1列目のマスを1にする場合, $x_{2,1}=1$に、\\
1行3列目のマスを1にしない場合, $x_{1,3}=0$とする.

$(N\times N)$の盤面におけるN-Queen場合の場合, 変数の組み合わせは次の様になる.\\
\begin{eqnarray*}
  \bm{x} &=&\{  \\
  &x_{1,1}&, x_{1,2}, \dots ,x_{1,N}\\
  &x_{2,1}&, x_{2,2}, \dots ,x_{2,N}\\
  \vdots  \\
  &x_{N,1}&, x_{N,2}, \dots ,x_{N,N}\}
\end{eqnarray*}

(ただしプログラムの実装時には, 配列の添字は0から$N-1$までの$N$次の配列になるので, このことを考慮する必要がある)

量子アニーリングで解くということは, 評価関数$f$の値が, この組み合わせ$\bm{x}$が正しいN-Queenの配置を構成できているときには小さい値になり, 間違った構成の場合には大きな値になる. その様なQUBO行列Qを決定していくことになる.

\subsection{縦横どの1列の数値の総和も等しい}

N-queenが正しい構成になるときには, 縦横のどの列についての総和も等しくなるという制約がある.

\begin{multline}
f_0(x)=\sum_k\sum_l\left(\sum_{(i,j)\in L_k} x_{i,j} - \sum_{(i,j)\in L_l} x_{i,j}\right)^2\\=\sum_k\sum_l\left(\sum_{(i_1,j_1)\in L_k}\sum_{(i_2,j_2)\in L_k} x_{i_1,j_1}x_{i_2,j_2}\right.\left.-2\sum_{(i_1,j_1)\in L_k}\sum_{(i_2,j_2)\in L_l} x_{i_1,j_1}x_{i_2,j_2}\right.\\\left.
+\sum_{(i_1,j_1)\in L_l}\sum_{(i_2,j_2)\in L_l} x_{i_1,j_1}x_{i_2,j_2}\right) \nonumber
\end{multline}

ただし,$L_k$は列に含まれるマスの集合のことで,以下の通り. $L_1 \sim L_{N}$は横の行, $L_{N+1} \sim L_{2N}$は縦の列, を表している.

\begin{eqnarray*}
L_1 &=& \{(1,1),(1,2),\dots,(1,N)\}\\
L_2 &=& \{(2,1),(2,2),\dots,(2,N)\}\\
\vdots\\
L_{N} &=& \{(N,1),(N,2),\dots,(N,N)\}\\
L_{N+1} &=& \{(1,1),(2,1),\dots,(N,1)\}\\
L_{N+2} &=& \{(1,2),(2,2),\dots,(N,2)\}\\
\vdots\\
L_{2N} &=& \{(1,N),(2,N),\dots,(N,N)\}
\end{eqnarray*}

また, ある列$k$の総和と, ある列$l$の総和の差を$d$とすると,
\[
\left(\sum_{(i,j)\in L_{k}} x_{i,j} - \sum_{(i,j)\in L_{l}}x_{i,j} \right)^2 = d^2
\]
であることから, $d=0$の時(すべての列の総和が等しい時)に最小になる.

(N-queenの各行および各列における総和Sは単に1になる.)

\subsubsection{各行(横方向)の値の総和は$S$である}

\begin{eqnarray*}
f_1(\bm{x})&=&\sum_i\left(\sum_jx_{i,j}-S\right)^2\\
&=&\sum_i\left(\sum_{j_1}\sum_{j_2}x_{i,j_1}x_{i,j_2} - 2 \sum_j S x_{i,j} + S^2\right)
\end{eqnarray*}

$x_{i,j}\in \{0,1\}$に注意すると, $x_{i,j}$は$x$を掛けた2次の形にしてもよい. また,
最小化問題では定数項は影響しないので無視できる.(以下同様に)

\subsubsection{各列(縦方向)の値の総和は$S$である}

\begin{eqnarray*}
f_2(\bm{x})&=&\sum_j\left(\sum_i x_{i,j}-S\right)^2\\
&=&\sum_j\left(\sum_{i_1}\sum_{i_2}x_{i_1,j}x_{i_2,j} - 2 \sum_i S x_{i,j} + S^2\right)
\end{eqnarray*}

\subsubsection{斜め方向の値の総和は$0$または$1$である}

\[
f_3(\bm{x}) = \left(\sum_d x_{d,d} - S \right)^2 + \left(\sum_d x_{d,N-d+1} - S \right)^2
\]

左側の項は右下がりの斜めの列に対する制約, 右側の項は右上がりの斜めの列に対する制約.

\subsection{盤面の数値の総和はNである}

8-queenの場合, $N$は$8$である.

\begin{eqnarray*}
f_4(\bm{x})&=&\left(\sum_i\sum_j x_{i,j} - N\right)^2=\left(\sum_i\sum_j x_{i,j} - N\right)\left(\sum_i\sum_j x_{i,j} - N\right)\\
&=&\sum_{i_1,i_2}\sum_{j_1,j_2}x_{i_1,j_1}x_{i_2,j_2}-2\sum_i\sum_j N x_{i,j} + N^2\\
&=&\sum_{i_1,i_2}\sum_{j_1,j_2}x_{i_1,j_1}x_{i_2,j_2} -2\sum_i\sum_j N x_{i,j}x_{i,j}
\end{eqnarray*}

$x_{i,j}\in \{0,1\}$を考慮して, 二次形式に直している.

%ある数値を丁度1回使う場合は、$\left(\sum_i\sum_j x_{i,j,n} - 1\right)^2=0$となり、1回も使わなかったり、複数回使ったりすると、0より大きな値になってしまうため、$1\sim N^2$を丁度1回使った場合に$f_4(x)$は最小になる

%\subsection{各マスには1つしか数字を入れてはならない}

%\[
%f_5(\bm{x})=\sum_i\sum_j\left( x_{i,j} - 1\right)^2
%\]

%あるマスに丁度1つの数値を入れる場合は、$\sum_i\sum_j\left(\sum_n x_{i,j,n} - 1\right)^2=0$となり、1つも入れなかったり、複数入れたりすると、0より大きな値になってしまうため、全てのマスに数値を丁度1つ入る時に$f_5(x)$は最小になる

\subsection{N-Queen生成の評価関数}

以上の制約を足し合わせることでN-Queenの生成に必要な評価関数を作ることができる.

\begin{eqnarray*}
  f(\bm{x}) &=& \lambda_1 f_1(\bm{x}) + \lambda_1 f_2(\bm{x}) + \lambda_2 f_3(\bm{x}) + \lambda_3 f_4(\bm{x})\\
  \\或いは、\\\\
  f(\bm{x}) &=& \lambda_1 f_0(\bm{x}) + \lambda_3 f_4(\bm{x})
\end{eqnarray*}

\section{実装}

\begin{verbatim}
pip install dwave-ocean-sdk
pip install openjij
\end{verbatim}

\subsection{初期化}

\begin{lstlisting}[language=Python]
from collections import defaultdict
import numpy as np
from dwave.optimization.symbols import Sum
from dwave.system import DWaveSampler, EmbeddingComposite
from openjij import SASampler, SQASampler
import time

rotate90 = lambda A: [list(x)[::-1] for x in zip(*A)]  # 90度回転用関数
transpose = lambda A: [list(x) for x in zip(*A)]  # 転置用関数

class EightQueen:
\end{lstlisting}

以下はEightQueenクラスにまとめてたもの. 必要なライブラリのインポートや変数,定数はパラメータを変更することで,N-Queenの大きさや使う量子アニーリングマシンなどを変更することができる.

\begin{lstlisting}[language=Python]
  def __init__(self):
    self.N = 8  # 盤面の大きさはNxN
    self.S = 1.0  # 各列の総和
    self.l1 = 2.0  # 罰金項の強さ(各行、各列の総和は等しい)
    self.l2 = 1.5  # 罰金項の強さ(斜めの総和は0または1)
    self.l3 = 1.0   # 罰金項の強さ(盤面上はN個)
    self.num_reads = 10000   # アニーリングを実行する回数
    self.token = 'XXXX'  # API token(個人のものを使用)
    self.solver = 'Advantage_system6.4'  # 量子アニーリングマシン
    self.machine = False
    Sampler = [SASampler(), SQASampler()]
    self.sampler = Sampler[0]
    self.ij_to_idx = {}

  def myindex(self):
    Q = defaultdict(lambda: 0)  # Q_i,j  (i, j)に入れる値
    # x_i,jの通し番号を記録
    ij_to_idx = {}
    idx = 0
    for i in range(self.N):
        for j in range(self.N):
            ij_to_idx[(i, j)] = idx
            idx += 1
    self.ij_to_idx = ij_to_idx
    return Q, ij_to_idx
\end{lstlisting}

\subsection{QUBO行列を作る}

各列の総和は等しい($f_1, f_2, f_3$の制約)

\begin{lstlisting}[language=Python]
  def constraint(self, Q, ij_to_idx):
    Q = self.constraint_row(Q, ij_to_idx)
    Q = self.constraint_clmn(Q, ij_to_idx)
    Q = self.constraint_diagonal(Q, ij_to_idx)
    Q = self.constraint_N(Q, ij_to_idx)
    return Q

  def constraint_row(self, Q,  ij_to_idx):
    # 各行の総和をSに制限(f1)
    for i in range(self.N):
        for j1 in range(self.N):
            for j2 in range(self.N):
                Q[(ij_to_idx[(i, j1)], ij_to_idx[(i, j2)])] += self.l1
            Q[(ij_to_idx[(i, j1)], ij_to_idx[(i, j1)])] -= 2 * self.S * self.l1
    return Q

  def constraint_clmn(self, Q,  ij_to_idx):
    # 各列の総和をSに制限(f2)
    for j in range(self.N):
        for i1 in range(self.N):
            for i2 in range(self.N):
                Q[(ij_to_idx[(i1, j)], ij_to_idx[(i2, j)])] += self.l1
            Q[(ij_to_idx[(i1, j)], ij_to_idx[(i1, j)])] -= 2 * self.S * self.l1
    return Q

  def constraint_diagonal(self, Q, ij_to_idx):
    # 斜めの各列の総和は0または1
    # 左上から右下の斜め制約 (i - j が等しい)
    for k in range(-self.N + 1, self.N):  # 有効なダイアゴナルの範囲
        diagonal_indices = [(i, j) for i in range(self.N) for j in range(self.N) if i - j == k]
        for idx1 in diagonal_indices:
            for idx2 in diagonal_indices:
                Q[(ij_to_idx[idx1], ij_to_idx[idx2])] += self.l2
            Q[(ij_to_idx[idx1], ij_to_idx[idx1])] -= 2 * self.S * self.l2

    # 右上から左下の斜め制約 (i + j が等しい)
    for k in range(2 * self.N - 1):  # 有効なダイアゴナルの範囲
        diagonal_indices = [(i, j) for i in range(self.N) for j in range(self.N) if i + j == k]
        for idx1 in diagonal_indices:
            for idx2 in diagonal_indices:
                Q[(ij_to_idx[idx1], ij_to_idx[idx2])] += self.l2
            Q[(ij_to_idx[idx1], ij_to_idx[idx1])] -= 2 * self.S * self.l2
    return Q

  def constraint_N(self, Q, ij_to_idx):
    # NxNの盤面上の数値の総和はNである
    for i1 in range(self.N):
        for j1 in range(self.N):
            for i2 in range(self.N):
                for j2 in range(self.N):
                    Q[(ij_to_idx[(i1, j1)], ij_to_idx[(i2, j2)])] += self.l3
            Q[(ij_to_idx[(i1, j1)], ij_to_idx[(i1, j1)])] -= 2 * self.N * self.l3
    return Q

\end{lstlisting}

\subsection{量子アニーリングの実行}

solver(Advantage\_system6.4)という量子アニーリングマシンで, num\_reads(今回は1000)回量子アニーリングを実行させることも可能だが有料になるので, Openjijのシミュレータを使うことにする. openjijからSASamplerまたはSQASamplerのいずれかを使って, ここまでに作成したQUBO行列Qに基づいた量子アニーリングの実行をシミュレートすることができる.

\begin{lstlisting}[language=Python]
  def annealing(self, Q):
    if self.machine:
        # 量子アニーリングの実行
        endpoint = 'https://cloud.dwavesys.com/sapi/'
        dw_sampler = DWaveSampler(solver=self.solver, token=self.token, endpoint=endpoint)
        sampler = EmbeddingComposite(dw_sampler)
        sampleset = sampler.sample_qubo(Q, num_reads=self.num_reads)
    else:
        # 焼きなまし法の実行
        sampler = self.sampler
        sampleset = sampler.sample_qubo(Q, num_reads=self.num_reads)
    return sampleset
\end{lstlisting}

\subsection{出力結果のフィルタ}

量子アニーリングが出力する結果には,最適化しきれずに制約条件を満たせないままのもの,あるいは同じ結果を重複して複数回出力しているもの,そして,回転対称や鏡像の盤面も複数含まれて出力されてくる.
まず制約条件を満たせないまま出力されているものを除外(check\_valid関数)し,対称な形も含めた重複出力を除外(check\_duplicate関数)する様にしている.

\begin{lstlisting}[language=Python]
  def gen_mat(self, sset):
    mat = []
    for i in range(self.N):
        w = []
        for j in range(self.N):
            w.append(sset[(self.ij_to_idx[(i, j)])])
        mat.append(w)
    return mat

  def same_p(self, sset1, sset2):
    s1 = np.array(sset1).flatten()
    s2 = np.array(sset2).flatten()
    if np.array_equal(s1, s2):
        return True
    return False

  def check_row_clmn(self, mat):
    Sum = self.S
    for i in range(self.N):
        s = 0.0
        for j in range(self.N):
            s += mat[i][j]
        if Sum < s:
            return False
    return True

  def check_diagonal(self, mat):   # def check2(mat):
    Sum = self.S
    # 左上から右下の斜め (i - j が等しい)
    for k in range(-self.N + 1, self.N):  # 有効なダイアゴナルの範囲
        diagonal_indices = [(i, j) for i in range(self.N) for j in range(self.N) if i - j == k]
        s = 0
        for d in diagonal_indices:
            s += mat[d[0]][d[1]]
        if Sum < s:
            return False
    # 右上から左下の斜め (i + j が等しい)
    for k in range(2 * self.N - 1):  # 有効なダイアゴナルの範囲
        diagonal_indices = [(i, j) for i in range(self.N) for j in range(self.N) if i + j == k]
        s = 0
        for d in diagonal_indices:
            s += mat[d[0]][d[1]]
        if Sum < s:
            return False
    return True

  def check_valid(self, sampleset):
    removelist = []
    for nth, sset in enumerate(sampleset):
        # Convert the solution into the 2D matrix form
        mat = self.gen_mat(sset)  # Properly generate the matrix form
        # Conduct various checks (rows, columns, diagonals)
        ck1 = self.check_row_clmn(mat)  # Check rows
        if not ck1:
            removelist.append(nth)
            continue
        ck2 = self.check_row_clmn(transpose(mat))  # Check columns
        if not ck2:
            removelist.append(nth)
            continue
        ck3 = self.check_diagonal(mat)  # Check diagonals
        if not ck3:
            removelist.append(nth)
            continue
    return self.remove_from_sampleset(sampleset, removelist)

  def check_duplicate(self, sampleset):
    # Ensure no duplicates due to symmetry or rotations
    removelist = []
    for nth1, sset1 in enumerate(sampleset):
        if nth1 in removelist:
            continue
        sets1 = self.variable_mat(sset1)
        for nth2, sset2 in enumerate(sampleset):
            if nth1==nth2 or nth2 in removelist:
                continue
            sets2 = self.variable_mat(sset2)
            for elem1 in sets1:
                for elem2 in sets2:
                    if self.same_p(elem1, elem2):
                        if nth2 not in removelist:
                            removelist.append(nth2)
                            break
            else:
                continue
    return self.remove_from_sampleset(sampleset, removelist)

  def remove_from_sampleset(self, sampleset, removelist):
    sampleset1 = []
    for nth, sset in enumerate(sampleset):
        if nth not in removelist:
            sampleset1.append(sset)
    return sampleset1

  def variable_mat(self, sset):
    sets = []
    set0 = self.gen_mat(sset)  # Convert to matrix form
    set1 = rotate90(set0)    # rotate 90
    set2 = rotate90(set1)   # rotate 180
    set3 = rotate90(set2)   # rotate 270
    set4 = np.fliplr(np.array(set)).copy().tolist()
    set5 = np.flipud(np.array(set)).copy().tolist()
    sets = [set0, set1, set2, set3, set4, set5]
    return sets

  def check(self, sampleset):
    sampleset1 = self.check_valid(sampleset)
    tm1 = time.time()
    print("valid   =", len(sampleset1))
    sampleset2 = self.check_duplicate(sampleset1)
    tm2 = time.time()
    print("checked =", len(sampleset2))
    return sampleset2, tm1, tm2
\end{lstlisting}

\subsection{結果出力(可視化)}

num\_reads回のアニーリングの結果から, 重複,回転対称,鏡像などのチェックをクリアしたものを可視化している.

\begin{lstlisting}[language=Python]
  def result(self, resultset):
    for nth, sset in enumerate(resultset):
        mat = self.gen_mat(sset)
        print(nth+1)
        for i in range(self.N):
            print(mat[i])
        print()
\end{lstlisting}

または,各出力結果ごとに,回転対称(90度,180度,270度)のもの,鏡像(左右,上下)のもの,を同時に生成出力して,確認できるようにしたものは以下.

\begin{lstlisting}[language=Python]
  def result_check_do(self, resultset):
    for nth, sset in enumerate(resultset):
        sset = self.gen_mat(sset)  # 行列の形に変換
        set1 = rotate90(sset)
        set2 = rotate90(set1)
        set3 = rotate90(set2)
        set4 = (np.array(sset)[:, ::-1]).tolist()
        set5 = (np.array(sset).T[:, ::-1].T).tolist()
        print(nth+1)
        for i in range(self.N):
            print(sset[i], set1[i], set2[i], set3[i], set4[i], set5[i])
        print()
\end{lstlisting}

\subsection{実行(主処理)}

\begin{lstlisting}[language=Python]
if __name__ == '__main__':
  start = time.time()
  eq = EightQueen()
  Q, ij2idx = eq.myindex()
  Q = eq.constraint(Q, ij2idx)
  ckpt1 = time.time()
  sampleset = eq.annealing(Q)
  ckpt2 = time.time()
  print("annealed=",len(sampleset))
  resultset, ckpt3, ckpt4 = eq.check(sampleset)
  eq.result_check_do(resultset)
  #
  print("Prepare:{}".format(ckpt1 - start))
  print("Annealing:{}".format(ckpt2 - ckpt1))
  print("ValidateCheck:{}".format(ckpt3 - ckpt2))
  print("DuplicateCheck:{}".format(ckpt4 - ckpt3))
  print("Total:{}".format(ckpt4 - start))
\end{lstlisting}

\section{まとめ}

実行例. 出力は以下の通り.
8クィーンの場合,10000回のnum\_readsの内で,制約条件を満たした出力だったのが7074個.その内,重複や対称な恰好のものを除外した結果,8クィーンの基本解である12個を全て出力できている.また,それ以上の数を出してくることはない(重複を総当たりで潰しているのですから).elapsed timeは,インテルシリコンのMacで1分弱で済んでいる.
num\_readsを小さくすると,出力はだんだん12個の基本解を網羅できなくなってくる.

\begin{verbatim}
  annealed= 10000
  valid   = 7074
  checked = 12
  1
  [0, 0, 0, 0, 1, 0, 0, 0]
  [0, 1, 0, 0, 0, 0, 0, 0]
  [0, 0, 0, 0, 0, 1, 0, 0]
  [1, 0, 0, 0, 0, 0, 0, 0]
  [0, 0, 0, 0, 0, 0, 1, 0]
  [0, 0, 0, 1, 0, 0, 0, 0]
  [0, 0, 0, 0, 0, 0, 0, 1]
  [0, 0, 1, 0, 0, 0, 0, 0]

  2
  [0, 0, 0, 1, 0, 0, 0, 0]
  [0, 0, 0, 0, 0, 1, 0, 0]
  [0, 0, 0, 0, 0, 0, 0, 1]
  [0, 1, 0, 0, 0, 0, 0, 0]
  [0, 0, 0, 0, 0, 0, 1, 0]
  [1, 0, 0, 0, 0, 0, 0, 0]
  [0, 0, 1, 0, 0, 0, 0, 0]
  [0, 0, 0, 0, 1, 0, 0, 0]

  3
  [0, 0, 0, 0, 0, 1, 0, 0]
  [0, 0, 0, 1, 0, 0, 0, 0]
  [0, 0, 0, 0, 0, 0, 1, 0]
  [1, 0, 0, 0, 0, 0, 0, 0]
  [0, 0, 1, 0, 0, 0, 0, 0]
  [0, 0, 0, 0, 1, 0, 0, 0]
  [0, 1, 0, 0, 0, 0, 0, 0]
  [0, 0, 0, 0, 0, 0, 0, 1]

  4
  [0, 0, 0, 0, 1, 0, 0, 0]
  [0, 0, 0, 0, 0, 0, 0, 1]
  [0, 0, 0, 1, 0, 0, 0, 0]
  [1, 0, 0, 0, 0, 0, 0, 0]
  [0, 0, 0, 0, 0, 0, 1, 0]
  [0, 1, 0, 0, 0, 0, 0, 0]
  [0, 0, 0, 0, 0, 1, 0, 0]
  [0, 0, 1, 0, 0, 0, 0, 0]

  5
  [0, 0, 0, 0, 0, 0, 0, 1]
  [0, 0, 0, 1, 0, 0, 0, 0]
  [1, 0, 0, 0, 0, 0, 0, 0]
  [0, 0, 1, 0, 0, 0, 0, 0]
  [0, 0, 0, 0, 0, 1, 0, 0]
  [0, 1, 0, 0, 0, 0, 0, 0]
  [0, 0, 0, 0, 0, 0, 1, 0]
  [0, 0, 0, 0, 1, 0, 0, 0]

  6
  [0, 0, 1, 0, 0, 0, 0, 0]
  [0, 0, 0, 0, 0, 1, 0, 0]
  [0, 0, 0, 1, 0, 0, 0, 0]
  [1, 0, 0, 0, 0, 0, 0, 0]
  [0, 0, 0, 0, 0, 0, 0, 1]
  [0, 0, 0, 0, 1, 0, 0, 0]
  [0, 0, 0, 0, 0, 0, 1, 0]
  [0, 1, 0, 0, 0, 0, 0, 0]

  7
  [0, 0, 0, 0, 1, 0, 0, 0]
  [0, 1, 0, 0, 0, 0, 0, 0]
  [0, 0, 0, 0, 0, 0, 0, 1]
  [1, 0, 0, 0, 0, 0, 0, 0]
  [0, 0, 0, 1, 0, 0, 0, 0]
  [0, 0, 0, 0, 0, 0, 1, 0]
  [0, 0, 1, 0, 0, 0, 0, 0]
  [0, 0, 0, 0, 0, 1, 0, 0]

  8
  [0, 0, 0, 1, 0, 0, 0, 0]
  [0, 1, 0, 0, 0, 0, 0, 0]
  [0, 0, 0, 0, 0, 0, 0, 1]
  [0, 0, 0, 0, 1, 0, 0, 0]
  [0, 0, 0, 0, 0, 0, 1, 0]
  [1, 0, 0, 0, 0, 0, 0, 0]
  [0, 0, 1, 0, 0, 0, 0, 0]
  [0, 0, 0, 0, 0, 1, 0, 0]

  9
  [0, 0, 0, 1, 0, 0, 0, 0]
  [0, 0, 0, 0, 0, 1, 0, 0]
  [1, 0, 0, 0, 0, 0, 0, 0]
  [0, 0, 0, 0, 1, 0, 0, 0]
  [0, 1, 0, 0, 0, 0, 0, 0]
  [0, 0, 0, 0, 0, 0, 0, 1]
  [0, 0, 1, 0, 0, 0, 0, 0]
  [0, 0, 0, 0, 0, 0, 1, 0]

  10
  [0, 0, 0, 0, 1, 0, 0, 0]
  [0, 1, 0, 0, 0, 0, 0, 0]
  [0, 0, 0, 1, 0, 0, 0, 0]
  [0, 0, 0, 0, 0, 1, 0, 0]
  [0, 0, 0, 0, 0, 0, 0, 1]
  [0, 0, 1, 0, 0, 0, 0, 0]
  [1, 0, 0, 0, 0, 0, 0, 0]
  [0, 0, 0, 0, 0, 0, 1, 0]

  11
  [0, 0, 1, 0, 0, 0, 0, 0]
  [0, 0, 0, 0, 0, 1, 0, 0]
  [0, 0, 0, 0, 0, 0, 0, 1]
  [1, 0, 0, 0, 0, 0, 0, 0]
  [0, 0, 0, 1, 0, 0, 0, 0]
  [0, 0, 0, 0, 0, 0, 1, 0]
  [0, 0, 0, 0, 1, 0, 0, 0]
  [0, 1, 0, 0, 0, 0, 0, 0]

  12
  [0, 0, 0, 0, 0, 1, 0, 0]
  [0, 1, 0, 0, 0, 0, 0, 0]
  [0, 0, 0, 0, 0, 0, 1, 0]
  [1, 0, 0, 0, 0, 0, 0, 0]
  [0, 0, 1, 0, 0, 0, 0, 0]
  [0, 0, 0, 0, 1, 0, 0, 0]
  [0, 0, 0, 0, 0, 0, 0, 1]
  [0, 0, 0, 1, 0, 0, 0, 0]

  Prepare:0.003168821334838867
  Annealing:20.62755823135376
  ValidateCheck:3.63771915435791
  DuplicateCheck:33.6769118309021
  Total:57.94535803794861

  プロセスは終了コード 0 で終了しました
\end{verbatim}


\end{document}